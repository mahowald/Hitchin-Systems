% Matthew Talk 2
\chapter{Hitchin Systems and Supersymmetric Field Theories}

\section{Introduction and definitions}

In previous talks, we've explored many properties of the Hitchin system
and its geometry, as well as several applications. The core idea of
this talk is that, given a certain supersymmetric field theory, we
can obtain the Hitchin system as a moduli space associated to this
theory via a standard procedure in quantum field theories. Gaiotto,
Moore, and Neitzke \cite{GMN} use this approach to construct a canonical
coordinate system on the Hitchin moduli space. The goal of this talk
is to explain some of the language of supersymmetric quantum field
theories, and describe how we can obtain $\mathcal{M}_{H}$ from a
particular class of such theories.


\subsection{Supersymmetry}

Consider $\mathbb{R}^{n}$. The \textbf{Poincare group} is the group
of translations and rotations of this space: $G=ISO(n)\cong\mathbb{R}^{n}\rtimes SO(n)$.
(In Minkowski signature, say for $\mathbb{R}^{n,1}$, we might write
$ISO(n,1)$ instead). Recall that there's a spin group defined as
a double cover of $SO(n)$:
\[
\xymatrix{1\ar[r] & \mathbb{Z}_{2}\ar[r] & Spin(n)\ar[r] & SO(n)\ar[r] & 1.}
\]
When we have such a double cover, we get an extension of the group
$G$ to the supergroup $\tilde{G}$. This gives an extension of the
Poincare algebra $\mathfrak{g}$ to the \textbf{super Poincare algebra}
$\tilde{\mathfrak{g}}$ coming from the double-cover of $SO(n)$.
$\tilde{\mathfrak{g}}$ is called the \textbf{supersymmetry algebra}.
Such $\tilde{g}$ are labelled by a choice of a representation of
$Spin(n)$. For irreducible representations of $Spin(n)$, there are
two possible cases:
\begin{itemize}
\item There is either a unique irreducible spinor representation $S$, and
any spinor representation has the form $S^{\oplus N}$ for some $N$,
or
\item There are two distinct irreducible real spinor representations $S_{+},S_{-}$,
and any spinor representation has the form $S_{+}^{\oplus N_{1}}\oplus S_{-}^{\oplus N_{2}}$
for some $N_{1},N_{2}$. (This occurs in dimensions $n\equiv2,6\mod8$).
\end{itemize}
Thus, when someone says $N=n$ or $N=\left(n_{1},n_{2}\right)$ supersymmetry,
they are specifying which extension of the Poincare algebra they're
referring to.

The supersymmetry Lie algebra $\tilde{\mathfrak{g}}$ splits into
an even and odd part:
\[
\tilde{\mathfrak{g}}=\mathfrak{g}^{0}\oplus\mathfrak{g}^{1},
\]
and is equipped with a skew-symmetric bracket $\left[\cdot,\cdot\right]:\tilde{\mathfrak{g}}\rightarrow\tilde{\mathfrak{g}}$
which satisfies the Jacobi identity. Note that $\left[\mathfrak{g}^{1},\mathfrak{g}^{1}\right]\subset\mathfrak{g}^{0}$,
$\mathfrak{g}^{0}=\mathfrak{g}$ (the Poincare algebra), and $\mathfrak{g}^{1}$
is just a linear representation of $\mathfrak{g}^{0}$.
\begin{example*}
For $N=2$ SUSY and $G=ISO(3,1)$, we'll be interested in a $\tilde{G}$
whose Lie algebra has even part
\[
\mathfrak{g}^{0}=\mbox{\ensuremath{\mathfrak{iso}}}(3,1)\oplus\mathbb{C}.
\]
The abelian $\mathbb{C}$ factor is central in $\tilde{\mathfrak{g}}$,
and has a canonical generator $Z$.
\end{example*}

\subsection{BPS States}

Let $\mathcal{H}$ denote the Hilbert space of states of our quantum
system on $\mathbb{R}^{3,1}$. We want to think of a state as being
labeled by a representation of $\tilde{G}$---the representation encodes
the data of the particle. For example, the vacuum state (``empty
space'') in $\mathcal{H}$ corresponds to the trivial representation
of $\tilde{G}$. The next simplest kind of state is one where space
is empty except for a single particle propogating with some definite
momentum $p\in T^{*}\mathbb{R}^{3,1}$. Call the subspace of the Hilbert
space consisting of single-particle states $\mathcal{H}^{1}$. Here,
there's some structure:
\begin{itemize}
\item $\mathcal{H}^{1}$ splits into components $\mathcal{H}_{M}^{1}$,
labeled by $M\in\mathbb{R}_{\geq0}$. $M^{2}$ is the eigenvalue of
a quadratic Casimir operator in $ISO(3,1)$. Physically, it's the
mass of the particle.
\item For $N=2$ SUSY as in our prior example, we also have a central generator
$Z\in\mathbb{C}$. Together, these give a decomposition
\[
\mathcal{H}^{1}=\bigoplus_{M,Z}\mathcal{H}_{M,Z}^{1}.
\]

\end{itemize}
Fix some momentum $p_{rest}\in\left(\mathbb{R}^{3,1}\right)^{*}$
with $||p_{rest}||^{2}=M^{2}$, and consider the subspace $\mathcal{H}_{M,Z}^{1,rest}$
on which the subgroup of translations along $\mathbb{R}^{3,1}$ acts
by $p_{rest}$. This is a representation of a subgroup $\tilde{G}_{rest}\subset\tilde{G}$
with
\[
\tilde{G}_{rest}=SO\left(3\right)\ltimes\tilde{T},
\]
where the ``super translation group'' $\tilde{T}$ is generated
by the ordinary translations $T=\mathbb{R}^{3,1}$ plus the central
character $Z$ and the ``odd translations'' $\mathfrak{g}^{1}$.
The odd translations act by a Clifford algebra (on an 8-dimensional
vector space). We can then count the number of unitary irreducible
representations of this Clifford algebra:
\begin{itemize}
\item If $M<\left|Z\right|$, then there are \emph{no} unitary representations
of the Clifford algebra.
\item If $M=\left|Z\right|$, the Clifford algebra is degenerate, and its
unique unitary irrep $S$ has dimension $2^{4/2}=4$.
\item If $M>\left|Z\right|$, the Clifford algebra is nondegenerate, and
its unique unitary irrep $S$ has dimension $2^{8/2}=16$.
\end{itemize}
States that satisfy $M=\left|Z\right|$ are called \textbf{BPS} (Bogomol'nyi,
Prasad, Sommerfield) \textbf{states}, and as one might expect, they
satisfy a set of differential equations depending on the field theory.
The BPS states are those in which half of the supersymmetry generators
are unbroken.


\subsection{Moduli of Vacua}

The ``moduli space'' associated to a QFT typically refers to the
moduli space of vacua. By \textbf{vacua}, we mean the quantum state
with the lowest possible energy.
\begin{description}
\item [{Question}] In what sense is the space of vacua a moduli space?
\end{description}
For scalar fields, these are labelled by the \textbf{vacuum expectation
value} (VEV). The VEV of an operator is (as the name suggests) the
expectation value of the operator in the vacuum (the quantum state
with the lowest possible energy). We can label a vacuum state by its
VEV, and this gives a moduli space of vacua.

For $N=2$ SUSY, the superalgebra has two representations with scalars:
\textbf{vectormultiplets} (one complex scalar), and \textbf{hypermultiplets}
(two complex scalars). This gives a local splitting of the moduli
of vacua $\mathcal{M}$ as
\[
\mathcal{M}=\mathcal{M}_{C}\oplus\mathcal{M}_{H},
\]
where $\mathcal{M}_{C}$ is the ``Coulomb branch'' (vectormultiplets),
and $\mathcal{M}_{H}$ is the ``Higgs branch'' (hypermultiplets).


\section{Compactification and Dimensional Reduction}

Compactification of a field theory is a process where, instead of
considering a general space $X$, we consider $X=M\times C$ where
$C$ is some compact space. Dimensional reduction is the limit of
the compactified theory where the volume of the compact space is shrunk
to zero, which produces an effective theory on the remaining dimensions.
\begin{example*}
[Toy Example] Consider a field theory on $X=\mathbb{R}^{n}\times S_{R}^{1}$
($S_{R}^{1}$ is the circle of radius $R$). Let $\theta$ be a coordinate
on $S_{R}^{1}$, and $x^{i}$ coordinates on $\mathbb{R}^{n}$. At
a fixed $x$ coordinate, the fields along the $S_{R}^{1}$ look like
\[
\phi|_{x}=\sum_{n}A_{n}\cos\left(\frac{2\pi n}{R}\theta\right)+B_{n}\sin\left(\frac{2\pi n}{R}\theta\right),
\]
where the coefficients $A_{n}$ and $B_{n}$ are determined by the
boundary conditions on $\phi$. As $R\rightarrow0$, the eigenvalues
$\lambda_{n}=\frac{2\pi n}{R}$ approach $\infty$, except for $n=0$.
Note that in quantum mechanics, $\hbar\lambda_{n}$ is the \emph{momentum}
of eigenstate $n$, so as $R\rightarrow\infty$, all mometums except
the trivial one also $\rightarrow\infty$. We should interpert $R\rightarrow0$
as meaning that, for finite energy (and hence, finite momentum), the
only eigenstate left is the trivial one.

If $\phi|_{x}$ is constant, it means that the field $\phi$ does
not depend on $\theta$---the dimensional reduction of the theory
on $S_{R}^{1}$ consists of the fields of the $\mathbb{R}^{n}\times S_{R}^{1}$
theory which do not depend on $\theta$.
\end{example*}
So, we have two equivalent perspectives on dimensional reduction to
$M$ of a theory on a space $M\times C$:
\begin{itemize}
\item It's the limit of the theory on $M\times C$ where the volume of $C$
contracts to zero, or
\item It's the theory on $M\times C$ where all fields are taken to be independent
of coordinates on $C$.
\end{itemize}

\begin{example*}
[Yang-Mills] We actually already encountered dimensional reduction
in one of the first lectures of the course. Consider classical Yang-Mills
theory.

Yang-Mills theory is a field theory defined for principal $G$-bundles
$P\rightarrow X$, where $X$ is a 4-dimensional Riemannian manifold.
\begin{description}
\item [{Fields}] Connections $A$ on $P$.
\item [{Lagrangian}]
\[
L\left(A\right)=\left|F_{A}\right|^{2}d\mu
\]

\end{description}
Recall that from the Lagrangian, we obtain the action functional by
\[
S\left(A\right)=\int_{X}L\left(A\right)=\int_{X}\left|F_{A}\right|^{2}d\mu.
\]
The equations of motion for Yang-Mills theory are
\[
d_{A}^{*}F_{A}=0,
\]
and the instantons are the (anti) self-dual connections:
\[
F_{A}=\pm*F_{A}.
\]
(Here, $*:\Omega_{X}^{2}\cong\Omega_{X}^{2}$ is the Hodge star operator).

In local coordinates we can write $d_{A}=d+A$, where
\[
A=A_{1}dx_{1}+A_{2}dx_{2}+A_{3}dx_{3}+A_{4}dx_{4}.
\]
Define
\[
F_{ij}:=\left[\nabla_{i},\nabla_{j}\right]=\frac{\partial}{\partial x_{i}}A_{j}-\frac{\partial}{\partial x_{j}}A_{i}+\left[A_{i},A_{j}\right].
\]
Look at the self-dual connections. Then, the instanton equation $F_{A}^{-}=0$
becomes
\begin{eqnarray*}
F_{12} & = & F_{34},\\
F_{13} & = & -F_{24},\\
F_{14} & = & F_{23}.
\end{eqnarray*}


Let's restrict attention to $X=C\times\mathbb{R}^{2}$, where $C$
is a Riemann surface. Suppose that we compactify and perform dimensional
reduction in $\mathbb{R}^{2}$ coordinates: restrict to $A_{j}$ which
are invariant under translation in $x_{3}$, $x_{4}$. Then, $A_{1}dx_{1}+A_{2}dx_{2}$
defines a connection on $C$. Relabel $A_{3}=\phi_{1}$ and $A_{4}=\phi_{2}$,
and define $\varphi=\phi_{1}-i\phi_{2}$; then the self-dual equations
become
\begin{eqnarray*}
F_{A}-\frac{1}{2}i\left[\varphi,\varphi^{*}\right] & = & 0,\\
\left[\nabla_{1}+i\nabla_{2},\varphi\right] & = & 0.
\end{eqnarray*}
If we think of $\varphi$ as defining a local section of $\Omega^{0}\left(C;ad(P)\otimes\mathbb{C}\right)$,
and set $\Phi=\frac{1}{2}\varphi dz\in\Omega^{1,0}\left(ad\left(P\right)\otimes\mathbb{C}\right)$
and $\Phi^{*}=\frac{1}{2}\varphi^{*}d\overline{z}\in\Omega^{0,1}\left(ad(P)\otimes\mathbb{C}\right)$,
then the equations become
\begin{eqnarray*}
F_{A}+\left[\Phi,\Phi^{*}\right] & = & 0,\\
\overline{\partial_{A}}\Phi & = & 0,
\end{eqnarray*}
the usual Hitchin equations.
\end{example*}

\section{5D Super Yang-Mills Theory}

Now, let's repeat the previous example, but including some supersymmetry.
5D Super Yang-Mills theory admits a conventional Lagrangian description:
Let $P$ be a principal $G$-bundle over $X$ (a 5-dimensional space).
The theory has:
\begin{description}
\item [{Fields}] Connections $A$ on $P$, sections $\phi^{i}$ of $Ad\left(P\right)$
($i=1,\ldots,5$), and fermions.
\item [{Lagrangian}]
\[
L=\frac{R}{8\pi^{2}}Tr\left[\frac{1}{R^{2}}F_{A}\wedge*F_{A}+\sum_{i=1}^{5}d_{A}\phi^{i}\wedge*d_{A}\phi^{i}+\mbox{fermions}\right].
\]
\end{description}
\begin{rem*}
~
\begin{enumerate}
\item 5D SYM is well-defined as an effective field theory, below a certain
energy scale. It is not obviously well-defined at arbitrarily high
energies.
\item There's this unusual $R$ factor appearing here that you should be
suspicious of. We'll explain where this comes from at the end of the
talk.
\end{enumerate}
\end{rem*}

\subsection*{Compactification on $C$}

Let's take $X=\mathbb{R}^{2,1}\times C$, where $C$ is a Riemann
surface. Analogous to the classical case, when we compactify 5D SYM
on $C$, we combine $\phi^{4}$ and $\phi^{5}$ into a complex-valued
1-form on $C$:
\[
\varphi=\left(\phi^{4}+i\phi^{5}\right)dz.
\]
Note that to be a sensible theory, we additionally require translation 
invariance along $\mathbb{R}^{2,1}$.
\begin{description}
\item [{Question}] What are the classical field configurations in the compactified
theory which preserve the supersymmetry? (Recall that these are the
BPS states!)
\end{description}
Assuming $\phi^{1},\phi^{2},\phi^{3}=0$, the equations satisfied
by the remaining fields are
\[
\tag{\ensuremath{\star}}\begin{cases}
 & F_{A}+R^{2}\left[\varphi,\varphi^{*}\right]=0,\\
 & \overline{\partial_{A}}\varphi=0,
\end{cases}
\]
which we recognize as (almost) Hitchin's equations. In other words,
the moduli space of vacua of $SYM[C]$ in the low energy limit is
\[
M_{C}\left[G\right]=\left\{ \mbox{solutions to }(\star)\right\} /\left\{ \mbox{gauge transformations}\right\} =\mathcal{M}_{H}.
\]

\begin{rem*}
We took $\phi^{1}=\phi^{2}=\phi^{3}=0$ above. If we don't, SUSY also
imposes equations on $\phi^{1},\phi^{2},\phi^{3}$:
\begin{align*}
d_{A}\phi^{i} & =0, & \left[\varphi,\phi^{i}\right] & =0, & \left[\phi^{i},\phi^{j}\right] & =0.
\end{align*}
But, at a generic point in the moduli space, these equations won't
have any nontrivial solutions, so the assumption that $\phi^{j}=0$
isn't much of an imposition.
\end{rem*}

A key difference between this example and dimensional reduction for classical
Yang-Mills theory is that we have dimensionally reduced to a theory on $\mathbb{R}^{2,1}$,
\emph{not} a theory on $C$. Instead of seeing Hitchin's moduli space as the moduli of
instantons for our theory, it appears as the moduli of BPS states!

The full moduli space of vacua has a Coulomb branch---identified with
the Hitchin moduli space---and Higgs branches attached to the specific
other points where nontrivial solutions for the $\phi^{j}$ exist.
(Unfortunate nomenclature: the moduli of Higgs bundles is the space
of solutions that live on the Coulomb branch...)


\section{Compactification from (2,0) 6D Theory}

Now let's talk about where that pesky $R$ factor came from. It turns
out that there's a famous 6D $N=\left(2,0\right)$ QFT. It doesn't
have a conventional Lagrangian description (or even a space of fields).
Instead, the inputs are a 6-dimensional manifold, together with a
Lie algebra $\mathfrak{g}$. Call this theory $X_{\mathfrak{g}}$.
It has the following properties:
\begin{itemize}
\item $X_{\mathfrak{g}}$ has $N=\left(2,0\right)$ SUSY in $d=6$.
\item $X_{\mathfrak{g}}$ has no parameters---no coupling constants or scale,
and the strength of the interaction can't be perturbed.
\item $X_{\mathfrak{g}}$ is conformally invariant.
\end{itemize}
Despite its unconventional description, we can still compactify $X_{\mathfrak{g}}$
to obtain lower-dimensional theories. In fact, 5D SYM is $X_{\mathfrak{g}}\left[S^{1}\right]$,
where the $R$ is the length of the $S^{1}$. So, $\mathcal{M}_{H}$
is obtained as the moduli space associated to $X_{\mathfrak{g}}\left[C\times S^{1}\right]$.
We could perform this compactification in either order: $\mathcal{M}_{H}$
can also be obtained as the moduli space associated to the theory
$X_{\mathfrak{g}}\left[C\right]$ compactified on $S^{1}$. \cite{GMN}
use this observation to produce canonical Darboux coordinate systems
on $\mathcal{M}_{H}$ and construct Calabi-Yau metrics in these coordinate
systems.

Some examples of information we can obtain from this perspective:
\begin{itemize}
\item Compactify $X_{\mathfrak{g}}$ on $C$ first to get a 4d $N=2$ supersymmetric
gauge theory with with Coulomb branch $\mathcal{B}$. Then, $\mathcal{B}$
is actually the Hitchin base, i.e., $\mathcal{M}_{H}\rightarrow\mathcal{B}$
with generic fiber a torus. Points $u\in\mathcal{B}$ correspond to
spectral curves $\Sigma_{u}\subset T^{*}C$, also known as ``Seiberg-Witten
curves.''
\item $\mathcal{M}_{H}$ is automatically hyperkahler because of supersymmetry.
\end{itemize}