%!TEX root = HitchinSystems.tex


\chapter{Introduction}

This course studies the Hitchin system from different perspectives and in various settings.

Hitchin studied a limiting form of the self-dual equations of Yang-Mills theory in four dimensions,
imposing invariance under two of the dimensions.  The resulting equations were found to be
conformally invariant with respect to the remaining two directions, so had a natural formulation
as equations on a Riemann surface.  The Hitchin moduli space, or moduli space of Higgs bundles,
will be defined as the space of solutions modulo gauge transformations.

Several basic properties of the space of solutions emerged.  First, the equations could be
interpreted formally as hyperkahler moment map equations for the compact gauge Group action
on the infinite-dimensional space of connections.  Since we quotient solutions by gauge
transformations, this means we have an interpretation of the Hithchin moduli space as a hyperkahler
quotient.  Thus it should inherit a hyperkaher structure.  Of the three independent complex structures,
the one most amenable to algebraic geometry is the ``Dolbeault'' one inherited from the complex
structue on the Riemann surface, $C$.  In that setting, the Hitchin moduli space has an
interpretation as stable Higgs pairs, i.e. pairs $(V,\varphi)$, where $V$ is a rank-$n$ holomorphic
vector bundle over $C$, and $\varphi\in \Gamma(End(V)\otimes K_C)$ is a holomorphic $n\times n$
matrix-valued one-form.  Stability requires that all $\varphi$-invariant sub-bundles have lesser slope
than $V$.  Finally, solutions are taken modulo holomorphic isomorphism of vector bundles.

Another structure which emerges in this complex structure
is that of an algebraically completely integrable system.
Recall first that a complete integrable system on a symplectic manifold (``phase space'') is a
maximal set of Poisson-commuting proper functions (``Hamiltonians").  Specifying the values
of these conserved quantities gives a common level set which is a Lagrangian torus.
This gives another definition:  a fibration by Lagrangian tori to a base manifold where the
Hamiltonians take their values.  The fiber map for the Hitchin space is defined by all the
invariant polynomials $Tr\varphi^k$.

The other complex structures are called ``de Rham,'' in which the moduli space is of flat
$GL_n(\mathbb C)$ connections modulo complex gauge transformations, and the ``Betti'' complex
structure in which the moduli space is irreducible representations of $\pi_1(X)$ modulo conjugation.
A variety of theorems relate these.  Betti and de Rham are equated by the Riemann-Hilbert
correspondence, in which one assigns to a flat connection the monodromy data around loops in the space.

\section{Local Case}

Consider that a holomorphic vector bundle has a local trivialization by a holomorphic frame,
so in the local setting we can assume $E = D\times \mathbb C^n,$ where either
$D$ is a disk $D = \{|z|<r\}\subset \mathbb C$ or $D = \mathbb C.$  In this frame, the
Higgs field looks like 

\section{Teichm\"uller theory}

\section{Link to Knots}

A hidden agenda for this seminar was to begin to explore the spectral curve from the perspective
of symplectic geometry.  Given a Riemann surface with complex structure $J$ and K\"ahler metric $h$,
it cotangent bundle $T^*C$ has a tautological holomorphic one-form $\theta$ and
holomorphic symplectic structure $\Omega = d\theta$.  
Treating $C$ as a real manifold, $T^*C$ has a canonical symplectic form $Re(\Omega)$
for which the spectral curve is Lagrangian, and we would like to consider it as an
object of the Fukaya category of $T^*C.$  In fact if we can equip $T^*C$ with a K\"ahler
form $\omega$ (this is not always possible, but is if $C = \mathbb C$) so that $T^*C$
is hyper-K\"ahler, then the spectral curve is special Lagrangian for the Calabi-Yau
form $\omega + i Im(\Omega),$ as $Im(\Omega)=0.$
To treat the spectral curve as an object in the Fukaya category of $T^*C$,
we equip it with a brane
structure and a flat vector bundle (local system).  It is often fruitful to consider
the case where the vector bundle is rank-one (so a $U(1)$ local system).

Algebraically, we can consider the family of spectral curves together with their
Jacobians.  This forms what is sometimes called the ``Beauville integrable system,"
in the sense that the Jacobians are tori fibered over a base.  Since these
spectral curves have an interpretation as special Lagrangians, we know that the
dimension of the space of their deformations is the first betti number --- same as the
dimension of the space of the fibers.

Let $L$ be the spectral curve and $\cL$ a local system on $L$, so
that $(L,\cL)$ is an object in the Fukaya category of $T^*C.$
(This involves a discrete choice of brane structure, which we ignore for present purposes.)
We consider the space $\cM(L,\cL)$ of (inequivalent) branes deformation equivalent to $(L,\cL)$.
The space $\cM(L,\cL)$
fibers over the space of Lagrangians deformation equivalent to $L$
modulo Hamiltonian deformations by the map which forgets the local system (and the brane
structure).  The tangent to this space at $L$ is closed modulo
exact one-forms, i.e. $H_{dR}^1(L).$
The fiber of the forgetful map is the space of $U(1)$ local systems on $L$, and the
tangent space to this space is $H^1(L,i\bR)$.
In fact, since nearby Lagrangians are all diffeomorphic, the fibration $\cM(L,\cL)$ has
a canonical horizontal distribution defined by ``keeping the local system constant,'' and
therefore the tangent space to $\cM(L,\cL)$ splits as a direct sum
$H^1(L,\bR)\oplus iH^1(L,\bR).$
In fact, we get more:  a local trivialization of $\cM(L,\cL)$ as a torus bundle.
This decomposition also suggests that we can ``absorb'' the geometric deformations
$H^1(L,\bR)$ by incorporating nonunitary line bundles.  If that were the case, then
a Lagrangian $L$ would generate a chart for $\cM(L,\cL)$ given by rank-one \emph{nonunitary}
local systems, a space we can identify with $(\bC^*)^{b_1(L)}.$  This will
turn out to be a symplectic
view of a cluster chart.






%
%Now $h$ induces a K\"ahler metric
%$\omega$ on $T^*C,$ and the triple $(\omega,Re\Omega,Im\Omega)$ define the
%hyper-K\"ahler structure.  A holomorphic curve $\Sigma \subset T^*C$ is calibrated
%by $\omega$ but is Lagrangian with respect to $Re\Omega.$  In this symplectic structure,
%we can consider $\Sigma$ as an object in the Fukaya category of $T^*C$.
%In fact is special Lagrangian
%with respect to the Calabi-Yau form $\omega - i Im \Omega.$

\section{Mirror Symmetry}

Recall that mirror symmetry is a kind of equivalence between a manifold $X$ and its mirror $Y$.
It goes deep, but in its simplest formulation, the Hodge diamonds of $X$ and $Y$ are related
by a mirror reflection.
The basic philosophy behind mirror symmetry is that mirror manifolds are dual torus fibrations,
according to the following reasoning.  A B-type D-brane (or B-brane) on a complex manifold is a coherent sheaf,
and we will only consider the case of the structure sheaf of
single point.  An A-type D-brane (or B-brane) in a symplectic
manifold is a special Lagrangian submanifold with a flat connection $\nabla$
--- and we shall focus on the line bundle case.

On a complex manifold $X$, the space of B-branes connected to a point brane is $X$ itself.
If there were a mirror manifold $Y$, the corresponding space of A-branes should again be $X,$
since the moduli spaces of mirror branes should correspond.  So
we should be able to find the mirror $X$ to $Y$ by identifying A-branes that correspond to point B-branes
and looking at their moduli.  By reversing the reasoning,
we also learn that the manifolds $X$ and $Y$ each are expressible
both as a space of A-branes.

If we fix an A-brane $(T,\nabla)$, the space of A-branes deformation equivalent
to (``connected to'') the brane $(T,\nabla)$ therefore has a forget map to the space
of special Lagrangians connected to $T$.  The fibers are tori $(S^1)^{b_1(T)}$ which encode
the monodromies, and it so happens
that by  linearizing
the special Lagrangian equations you can see that $b_1(T)$ is also the dimension of the space
of special Lagrangians at $T.$  So the D-brane moduli space is an integrable system, a torus
fibration over special Lagrangian moduli space (Lagrangianicity must be argued separately).
Now $Y$, like $X,$ should itself be some an A-brane (on $X$) moduli space, so has a torus
fibration structure, a map $Y\to B$.  What if we took $T$ to be a torus fiber?
Then the dimensions are right, since we want $b_1(T)$ to be equal to the complex dimension
of $X$ and $Y,$ equivalently the real dimension of $T$.  A fiber also has the property that it is disjoing from its deformations -- just like
its dual point B-branes.  If this is the case (and there are other arguments supporting this which
we ignore), then the D-brane moduli space of a torus fiber $T$ fibers over the same base $B$ (its geometric
deformations), with the fiber over $T_b = \pi^{-1}(b)$ being the flat line bundles on $T_b,$ i.e. the
dual torus.  Therefore

$$\emph{Mirror pairs are dual Lagrangian torus fibrations}$$

The geometric Langlands program is a connection between Hitchin moduli spaces of
Langlands dual groups $G$ and $G^\vee.$
Each gives rise to an integrable system via its spectral curve construction.
Hausel-Thaddeus found these spaces to be dual torus fibrations
and computed the (stringy) Hodge numbers of both sides, revealing them to be mirror.


