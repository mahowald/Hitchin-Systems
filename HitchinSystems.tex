\documentclass[oneside,english]{amsbook}
\usepackage{lmodern}
\renewcommand{\sfdefault}{lmss}
\renewcommand{\ttdefault}{lmtt}
\renewcommand{\familydefault}{\rmdefault}
\usepackage[T1]{fontenc}
\usepackage[latin9]{inputenc}
\usepackage{amsthm}

\makeatletter
%%%%%%%%%%%%%%%%%%%%%%%%%%%%%% Textclass specific LaTeX commands.
\numberwithin{section}{chapter}
\numberwithin{equation}{section}
\numberwithin{figure}{section}

\makeatother

\usepackage{babel}
\begin{document}

\title{Hitchin Systems Seminar}

\maketitle
\tableofcontents{}


\chapter{Overview and Definitions}


\chapter{Symplectic Quotients in Finite and Infinite Dimensions}


\chapter{Hyperkahler Quotients and Integrable Systems}


\chapter{Stability and the Narasimhan-Seshadri Theorem}


\chapter{Teichmueller Theory}


%% Matthew's talk #1 
\chapter{Construction of the Hitchin Moduli Space}


This talk closely follows \cite{H1}. In this chapter, it is shown
that the Hitchin moduli space $\mathcal{M}_{H}$ is a smooth manifold
of dimension $12(g-1)$. We also prove that $\mathcal{M}_{H}$ is
equipped with a complete hyperkahler metric. 


\section{Definitions}

Let $M$ be a Riemannian surface, $P\rightarrow M$ a principal $G=SO\left(3\right)$-bundle,
$V$ the associated rank 2 vector bundle, and $\mathcal{G}$ the group
of gauge transformations ($\mathcal{G}=Map\left(M,G\right)$). Recall
that for a connection $A$ on $P$ and $\Phi\in\Omega_{M}^{1,0}\left(adP\otimes\mathbb{C}\right)$,
the self-dual equations are 
\begin{align*}
\tag{\ensuremath{\star}}F_{A}+\left[\Phi,\Phi^{*}\right] & =0,\\
\overline{\partial_{A}}\Phi & =0
\end{align*}


For a connection $A$, $u\in\mathcal{G}$ acts by 
\[
u\left(A\right)=uAu^{-1}-\left(du\right)u^{-1}
\]
(where the covariant derivative is $\nabla_{A}=d+A\wedge$, i.e.,
\[
\nabla_{u\left(A\right)}s=u\nabla_{A}\left(u^{-1}s\right)
\]
for a section $s$ of $V$.

For $\Phi\in\Omega_{M}^{1,0}\left(adP\otimes\mathbb{C}\right)$, $u$
acts by 
\[
u\left(\Phi\right)=u\Phi u^{-1},
\]
where we regard $adP\otimes\mathbb{C}$ as the bundle of trace-zero
endomorphisms of $V$.
\begin{defn}
The Hitchin moduli space is 
\[
\mathcal{M}_{H}:=\left\{ \mbox{solutions to (}\star\mbox{)}\right\} /\mathcal{G}.
\]
\end{defn}
\begin{description}
\item [{Goal}] To learn about the geometry and topology of $\mathcal{M}_{H}$.
\end{description}

\section{Summary of results}

Let $V$ be a rank 2, odd degree vector bundle over a Riemannian surface
$M$. Then, 
\begin{itemize}
\item $\mathcal{M}_{H}$ is a smooth manifold of dimension $12\left(g-1\right)$.
\item $\mathcal{M}_{H}$ has a natural metric.
\item $\mathcal{M}_{H}$'s metric is complete and hyperkahler (in fact,
$\mathcal{M}_{H}$ is a hyperkahler quotient).\end{itemize}
\begin{rem*}
$V$ has odd degree $\implies$ there are no reducible solutions to
$\left(\star\right)$ $\implies$ $\mathcal{G}$ acts freely on $\left(\star\right)$.
\end{rem*}

\section{$\mathcal{M}_{H}$ is a dimension $12\left(g-1\right)$ smooth manifold}

First, to get the expected dimension, we compute the dimension of
the tangent space to $\mathcal{M}_{H}$ at a regular point (i.e.,
one with trivial isotropy group). 


\subsection{Idea of Proof}
\begin{itemize}
\item Linearize $(\star)$ to determine the expected dimension.
\item Let $\left(A\times\Omega\right)_{0}$ denote ``regular points,''
i.e., ones which are only fixed by the identity in $\mathcal{G}$
(those with trivial isotropy group). Then, exhibit $\mathcal{M}_{H}$
is a smooth submanifold of $\left(\mathcal{A}\times\Omega\right)_{0}/\mathcal{G}$
using the regular value theorem.
\end{itemize}

\subsection{Linearization of $(\star)$}

Let $\left(\dot{A},\dot{\Phi}\right)\in\Omega_{M}^{1}\left(adP\right)\oplus\Omega_{M}^{1,0}\left(adP\otimes\mathbb{C}\right)$.
To get the linearization of $(\star$), fix a base point $\left(A,\Phi\right)$
and look at 
\[
\left.\frac{d}{dt}\right|_{t=0}\left(\star\right)\left(A+t\dot{A},\Phi+t\dot{\Phi}\right).
\]
Recalling that $F_{A}=dA+A\wedge A$, the first equation becomes 
\[
d_{A}\dot{A}+\left[\dot{\Phi},\Phi^{*}\right]+\left[\Phi,\dot{\Phi}^{*}\right]=0,
\]
and the second is 
\[
\overline{\partial_{A}}\dot{\Phi}+\dot{A}^{0,1}\wedge\Phi=0.
\]


$\left(\dot{A},\dot{\Phi}\right)$ arises from an infinitesimal gauge
transformation $\dot{\psi}\in\Omega_{M}^{0}\left(adP\right)$ if 
\begin{align*}
\dot{A} & =d_{A}\dot{\psi}, & \dot{\Phi} & =\left[\Phi,\dot{\psi}\right].
\end{align*}
Let 
\begin{eqnarray*}
d_{1}:\Omega_{M}^{0}\left(adP\right) & \longrightarrow & \Omega_{M}^{1}\left(adP\right)\oplus\Omega_{M}^{1,0}\left(adP\otimes\mathbb{C}\right)\\
\dot{\psi} & \mapsto & \left(d_{A}\dot{\psi},\left[\Phi,\dot{\psi}\right]\right)
\end{eqnarray*}
and 
\begin{eqnarray*}
d_{2}:\Omega_{M}^{1}\left(adP\right)\oplus\Omega_{M}^{1,0}\left(adP\otimes\mathbb{C}\right) & \longrightarrow & \Omega_{M}^{2}\left(adP\right)\oplus\Omega_{M}^{2}\left(adP\otimes\mathbb{C}\right)\\
\left(\dot{A},\dot{\Phi}\right) & \mapsto & \left(d_{A}\dot{A}+\left[\dot{\Phi},\Phi^{*}\right]+\left[\Phi,\dot{\Phi}^{*}\right],\overline{\partial_{A}}\dot{\Phi}+\dot{A}^{0,1}\wedge\Phi\right).
\end{eqnarray*}
Then, $d_{1},d_{2}$ define an \emph{elliptic complex} with index
$3\left(2-2g\right)-6\left(g-1\right)=12\left(1-g\right)$. The Atiyah-Singer
index theorem then says that 
\[
\dim H^{0}-\dim H^{1}+\dim H^{2}=12\left(1-g\right).
\]



\subsection{Elliptic Complexes}

Now, a short digression into elliptic complexes. Let $X$ be a manifold,
$\pi:T^{*}X\rightarrow X$ be projection onto the base, and $\left\{ E_{k}\rightarrow X\right\} $
a collection of vector bundles over $X$. 
\begin{defn}
A chain complex 
\[
\xymatrix{\cdots\ar[r] & \Gamma\left(E_{k-1}\right)\ar[r]^{P_{k-1}} & \Gamma\left(E_{k}\right)\ar[r]^{P_{k}} & \Gamma\left(E_{k+1}\right)\ar[r] & \cdots}
\]
is \textbf{elliptic} if the corresponding sequence of symbols 
\[
\xymatrix{\cdots\ar[r] & \pi^{*}E_{k-1}\ar[r]^{\sigma\left(P_{k-1}\right)} & \pi^{*}E_{k}\ar[r]^{\sigma\left(P_{k}\right)} & \pi^{*}E_{k+1}\ar[r] & \cdots}
\]
is exact.\end{defn}
\begin{example*}
Let $P:E_{1}\rightarrow E_{2}$ be a differential operator. Then,
the complex 
\[
\xymatrix{0\ar[r] & \Gamma\left(E_{1}\right)\ar[r]^{P} & \Gamma\left(E_{2}\right)\ar[r] & 0}
\]
is elliptic means that 
\[
\xymatrix{0\ar[r] & \pi^{*}E_{1}\ar[r]^{\sigma\left(P\right)} & \pi^{*}E_{2}\ar[r] & 0}
\]
is exact, i.e., that $\sigma\left(P\right):\pi^{*}E_{1}\rightarrow\pi^{*}E_{2}$
is an isomorphism. Recall that this is the ``usual'' definition
of elliptic differential operator. 
\end{example*}
Now, return to our elliptic complex. 

\[
\xymatrix{\Omega_{M}^{0}\left(adP\right)\ar[r]^-{d_{1}} & \Omega_{M}^{1}\left(adP\right)\oplus\Omega_{M}^{1,0}\left(adP\otimes\mathbb{C}\right)\ar[r]^-{d_{2}} & \Omega_{M}^{2}\left(adP\right)\oplus\Omega_{M}^{2}\left(adP\otimes\mathbb{C}\right)\\
\dot{\psi}\ar@{|->}[r] & \left(d_{A}\dot{\psi},\left[\Phi,\dot{\psi}\right]\right)\\
 & \left(\dot{A},\dot{\Phi}\right)\ar@{|->}[r] & \left(d_{A}\dot{A}+\left[\dot{\Phi},\Phi^{*}\right]+\left[\Phi,\dot{\Phi}^{*}\right],\overline{\partial_{A}}\dot{\Phi}+\dot{A}^{0,1}\wedge\Phi\right)
}
\] 

By construction, we're interested in $H^{1}$ of this complex: $H^{0}=\ker d_{1}$
is the covariantly constant $\dot{\psi}$ which commute with $\Phi$.
These correspond to reducible solutions to $(\star$). So, if $H^{0}\neq0$,
then $\left(A,\Phi\right)$ is reducible. In fact, by considering
$d_{2}^{*}$, we can show that $H^{2}=0$ as well \cite{H1}. Hence,
\[
-\dim H^{1}=12\left(1-g\right),
\]
i.e., for a regular point $\left(A,\Phi\right)$, 
\[
\boxed{\dim T_{\left(A,\Phi\right)}\mathcal{M}_{H}=12\left(g-1\right)}.
\]



\subsection{$\mathcal{M}_{H}$ is a smooth manifold}

This is a sketch, for complete details see \cite{H1}. 
\begin{defn}
$\left(A,\Phi\right)$ is a \textbf{regular point} if the isotropy
group of $\left(A,\Phi\right)$ is the identity (i.e., there are no
nontrivial gauge transformations fixing this point).
\end{defn}
Recall that infinitesimally, these are the points where there are
no nonzero solutions to $d_{1}\dot{\psi}=0$. 

Let $\left(\mathcal{A}\times\Omega\right)_{0}$ denote the open set
of regular points in $\mathcal{A}\times\Omega$, and 
\[
B:=\left(\mathcal{A}\times\Omega\right)_{0}/\mathcal{G}.
\]
By construction, $B$ is a Banach manifold with the quotient topology.
We have the map 
\[
d_{1}^{*}:\Omega_{M}^{1}\left(adP\right)\oplus\Omega_{M}^{1,0}\left(adP\otimes\mathbb{C}\right)\longrightarrow\Omega_{M}^{0}\left(adP\right).
\]
Define a \emph{slice} of $\mathcal{M}_{H}$ to be $\ker d_{1}^{*}$,
at some fixed $\left(A_{0},\Phi_{0}\right)$---then, the slices provide
coordinate patchs for $B$. Set 
\begin{eqnarray*}
f:\ker d_{1}^{*} & \longrightarrow & \Omega_{M}^{2}\left(adP\right)\oplus\Omega_{M}^{2}\left(adP\otimes\mathbb{C}\right)\\
\left(A,\Phi\right) & \mapsto & \left(F_{A}+\left[\Phi,\Phi^{*}\right],\overline{\partial_{A}}\Phi\right);
\end{eqnarray*}
then, $f^{-1}\left(0,0\right)$ is a smooth submanifold of $\ker d_{1}^{*}$
with dimension $12(g-1)$. 

Since $\ker d_{1}^{*}$ form coordinate patches for $B$, the remainder
of the proof is just arguing that the $\ker d_{1}^{*}$ patch together
to form a smooth manifold. 


\section{The tangent space}

Thanks to our results in the previous section, we have an explicit
description of the tangent space to $\mathcal{M}_{H}$: 
\[
T_{\left(A,\Phi\right)}\mathcal{M}_{H}=\left\{ \left(\dot{A},\dot{\Phi}\right)\left|\begin{matrix}d_{A}\dot{A}+\left[\dot{\Phi},\Phi^{*}\right]+\left[\Phi,\dot{\Phi}^{*}\right]=0,\\
\overline{\partial_{A}}\dot{\Phi}+\dot{A}^{0,1}\wedge\Phi=0,\\
d_{A}^{*}\dot{A}+Re\left[\Phi^{*},\dot{\Phi}\right]=0.
\end{matrix}\right.\right\} 
\]
The third equation appears because $d_{1}^{*}\left(\dot{A},\dot{\Phi}\right)=d_{A}^{*}\dot{A}+Re\left[\Phi^{*},\dot{\Phi}\right]$. 


\section{$\mathcal{M}_{H}$ has a complete hyperkahler metric}


\section{Summary of topological results}


\chapter{Integrable Systems and Spectral Curves}


\chapter{Meromorphic Connections and Stokes Data}


\chapter{The Langlands Program and Relations to Geometric Langlands}


\chapter{Spectral Curves and Irregular Singularities}


\chapter{The Stokes Groupoid}


\chapter{Cluster Varieties}


\chapter{The Hitchin System and Teichmueller Theory II}


\chapter{Geometric Langlands and Mirror Symmetry}
\begin{thebibliography}{BNR}
\bibitem[AB]{AB}M.~F.~Atiyah and R.~Bott, The Yang-Mills equations
over Riemann surfaces. Philos.~Trans.~Roy.~Soc. London A 308 (1982)
523--615.

\bibitem[B1]{B1}P.~Boalch, ``Geometry and Braiding of Stokes Data;
Fission and Wild Character Varieties,'' Annals of Math.~\textbf{179}
(2014) 301--365. http://annals.math.princeton.edu/wp-content/uploads/annals-v179-n1-p05-s.pdf 

\bibitem[B2]{B2}P.~Boalch, ``Hyperkaehler Manifolds and Nonabelian
Hodge Theory of (Irregular) Curves,'' arXiv:1203.6607

\bibitem[BB]{BB}O.~Biquard and P.~Boalch. Wild non-abelian Hodge
theory on curves - 2004, Compos.~Math.~140 (2004) 179--204.

\bibitem[BNR]{BNR}A.~Beauville, M.~S.~Narasimhan, and S.~Ramanan,
``Spectral curves and the generalized theta divisor,'' J.~reine
angew.~Math. 398 (1989) 169--179 URL: http://math1.unice.fr/\textasciitilde{}beauvill/pubs/bnr.pdf

\bibitem[D]{D}S.~K.~Donaldson, ``A new proof of a theorem of Narasimhan
and Seshadri,'' J.~Differential Geometry, \textbf{18} (1983) 269--277.

\bibitem[DM]{DM}R.~Donagi and E.~Markman, ``Spectral covers, algebraically
completely integrable Hamiltonian systems, and moduli of bundles.''
arXiv:alg-geom/9507017

\bibitem[FG]{FG}V.~Fock and A.~Goncharov, Moduli spaces of local
systems and higher Teichmueller theory, Publ. Math. Inst. Hautes Etudes
Sci. No. 103 (2006), 1\textendash{}211. 

\bibitem[GMN]{GMN} D.~Giaotto, G.~Moore, and A.~Neitzke, Wall-crossing,
Hitchin systems, and the WKB approximation. arXiv:0907.3987

\bibitem[H1]{H1}N.~Hitchin, ``The self-duality equations on a Riemann
surface,'' Proc. Long. Math. Soc. 55 (1987) 59--126. 

\bibitem[H2]{H2}N.~Hitchin, ``Stable bundles and integrable systems,''
Duke Math J. 54 (1987) 91--114. 

\bibitem[H3]{H3}N.~Hitchin, ``Lie Groups and Teichmueller Space,''
Topology 31 (1992) 449--473. 

\bibitem[HT]{HT}T.~Hausel and M.~Thaddeus. Mirror symmetry, Langlands
duality, and the Hitchin system. Invent. Math., 153 (1):197\textendash{}229,
2003. 

\bibitem[KW]{KW}A.~Kapustin and E.~Witten, ``Electric-Magnetic
Duality and the Geometric Langlands Program,'' CNTP 1 (2007()1--236.
arXiv:hep-th/0604151

\bibitem[N]{N}A.~Neitzke, ``Hitchin systems in N=2 field theory,''
https://www.ma.utexas.edu/users/neitzke/expos/hitchin-systems.pdf 

\bibitem[NS]{NS}M.~S.~Narasimhan and C.~S.~Seshadri, `Stable
and unitary vector bundles on a compact Riemann surface,' Ann.~of
Math.~\textbf{82} (1965) 540--567.

\bibitem[W]{W}E.~Witten, ``Gauge Theory and Wild Ramification,''
arXiv:0710.0631\end{thebibliography}

\end{document}
