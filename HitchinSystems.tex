\documentclass[oneside,english]{amsbook}
\usepackage{lmodern}
\renewcommand{\sfdefault}{lmss}
\renewcommand{\ttdefault}{lmtt}
\renewcommand{\familydefault}{\rmdefault}
\usepackage[T1]{fontenc}
\usepackage[latin9]{inputenc}
\usepackage{amsthm}
\usepackage{amssymb}
\usepackage{esint}
\usepackage[all]{xy}
\usepackage{mathrsfs}
\usepackage{stmaryrd}
\usepackage{enumerate,tikz}



\usepackage{anysize}
\marginsize{1in}{1in}{1in}{1in}

\usepackage{amsmath,amsthm,amssymb,amsbsy,amstext,amsopn,amsfonts,bbm,psfrag,color,float}
\usepackage{graphicx}
\usepackage{latexsym}
\usepackage[all]{xy}

\usepackage{booktabs}
\usepackage{multirow}

\usepackage{tikz}
\usetikzlibrary{matrix}

\usepackage[bookmarks=true, bookmarksopen=true,%
bookmarksdepth=3,bookmarksopenlevel=2,%
colorlinks=true,%
linkcolor=blue,%
citecolor=blue,%
filecolor=blue,%
menucolor=blue,%
urlcolor=blue]{hyperref}

\newenvironment{HW}{\color{violet}{\bf HW:} \footnotesize}{}
\newenvironment{VS}{\color{blue}{\bf VS:} \footnotesize}{}
\newenvironment{EZ}{\color{red}{\bf EZ:} \footnotesize}{}
\newenvironment{DT}{\color{magenta}{\bf DT:}\footnotesize}{}


\newtheorem{lemma}{Lemma}[section]
\newtheorem{proposition}[lemma]{Proposition}
\newtheorem{corollary}[lemma]{Corollary}
\newtheorem{theorem}[lemma]{Theorem}

\theoremstyle{definition}
\newtheorem{definition}[lemma]{Definition}
\newtheorem{remark}[lemma]{Remark}
\newtheorem{example}[lemma]{Example}
\newtheorem{conjecture}[lemma]{Conjecture}
\newtheorem{question}[lemma]{Question}

% font shorts

\newcommand{\A}{\mathbb{A}}
\newcommand{\B}{\mathbb{B}}
\newcommand{\C}{\mathbb{C}}
\newcommand{\D}{\mathbb{D}}
\newcommand{\E}{\mathbb{E}}
\newcommand{\F}{\mathbb{F}}
\newcommand{\G}{\mathbb{G}}
\renewcommand{\H}{\mathbb{H}}
\newcommand{\I}{\mathbb{I}}
\newcommand{\J}{\mathbb{J}}
\newcommand{\K}{\mathbb{K}}
\renewcommand{\L}{\mathbb{L}}
\newcommand{\M}{\mathbb{M}}
\newcommand{\N}{\mathbb{N}}
\renewcommand{\O}{\mathbb{O}}
\renewcommand{\P}{\mathbb{P}}
\newcommand{\Q}{\mathbb{Q}}
\newcommand{\R}{\mathbb{R}}
\newcommand{\T}{\mathbb{T}}
\newcommand{\U}{\mathbb{U}}
\newcommand{\V}{\mathbb{V}}
\newcommand{\W}{\mathbb{W}}
\newcommand{\X}{\mathbb{X}}
\newcommand{\Y}{\mathbb{Y}}
\newcommand{\Z}{\mathbb{Z}}

\newcommand{\cA}{\mathcal{A}}
\newcommand{\cB}{\mathcal{B}}
%\newcommand{\cC}{\mathcal{C}}
\newcommand{\cD}{\mathcal{D}}
\newcommand{\cE}{\mathcal{E}}
\newcommand{\cF}{\mathcal{F}}
\newcommand{\cG}{\mathcal{G}}
\newcommand{\cH}{\mathcal{H}}
\newcommand{\cI}{\mathcal{I}}
\newcommand{\cJ}{\mathcal{J}}
\newcommand{\cK}{\mathcal{K}}
\newcommand{\cL}{\mathcal{L}}
\newcommand{\cM}{\mathcal{M}}
\newcommand{\cN}{\mathcal{N}}
\newcommand{\cO}{\mathcal{O}}
\newcommand{\cP}{\mathcal{P}}
\newcommand{\cQ}{\mathcal{Q}}
\newcommand{\cR}{\mathcal{R}}
\newcommand{\cS}{\mathcal{S}}
\newcommand{\cT}{\mathcal{T}}
\newcommand{\cU}{\mathcal{U}}
\newcommand{\cV}{\mathcal{V}}
\newcommand{\cW}{\mathcal{W}}
\newcommand{\cX}{\mathcal{X}}
\newcommand{\cY}{\mathcal{Y}}
\newcommand{\cZ}{\mathcal{Z}}

\newcommand{\bA}{\mathbf{A}}
\newcommand{\bB}{\mathbf{B}}
\newcommand{\bC}{\mathbf{C}}
\newcommand{\bD}{\mathbf{D}}
\newcommand{\bE}{\mathbf{E}}
\newcommand{\bF}{\mathbf{F}}
\newcommand{\bG}{\mathbf{G}}
\newcommand{\bH}{\mathbf{H}}
\newcommand{\bI}{\mathbf{I}}
\newcommand{\bJ}{\mathbf{J}}
\newcommand{\bK}{\mathbf{K}}
\newcommand{\bL}{\mathbf{L}}
\newcommand{\bM}{\mathbf{M}}
\newcommand{\bN}{\mathbf{N}}
\newcommand{\bO}{\mathbf{O}}
\newcommand{\bP}{\mathbf{P}}
\newcommand{\bQ}{\mathbf{Q}}
\newcommand{\bR}{\mathbf{R}}
\newcommand{\bS}{\mathbf{S}}
\newcommand{\bT}{\mathbf{T}}
\newcommand{\bU}{\mathbf{U}}
\newcommand{\bV}{\mathbf{V}}
\newcommand{\bW}{\mathbf{W}}
\newcommand{\bX}{\mathbf{X}}
\newcommand{\bY}{\mathbf{Y}}
%\newcommand{\bZ}{\mathbf{Z}}

\newcommand{\coeffs}{\mathbbm{k}}
\newcommand{\e}{\epsilon}
\newcommand{\red}[1]{{\color{red} #1}}



\makeatletter
%%%%%%%%%%%%%%%%%%%%%%%%%%%%%% Textclass specific LaTeX commands.
\numberwithin{section}{chapter}
\numberwithin{equation}{section}
\numberwithin{figure}{section}
\theoremstyle{plain}
\newtheorem{thm}{\protect\theoremname}
  \theoremstyle{definition}
  \newtheorem{defn}[thm]{\protect\definitionname}
  \theoremstyle{remark}
  \newtheorem*{rem*}{\protect\remarkname}
  \theoremstyle{definition}
  \newtheorem*{example*}{\protect\examplename}
 \theoremstyle{definition}
 \newtheorem*{defn*}{\protect\definitionname}
  \theoremstyle{plain}
  \newtheorem*{cor*}{\protect\corollaryname}
  \theoremstyle{plain}
  \newtheorem{prop}{\protect\theoremname}
\newtheorem{conj}[thm]{Conjecture}


\makeatother

\usepackage{babel}
  \providecommand{\corollaryname}{Corollary}
  \providecommand{\definitionname}{Definition}
  \providecommand{\examplename}{Example}
  \providecommand{\remarkname}{Remark}
\providecommand{\theoremname}{Theorem}

\begin{document}

\title{Hitchin Systems Seminar}

\maketitle
\tableofcontents{}


%!TEX root = HitchinSystems.tex


\chapter{Introduction}

This course studies the Hitchin system from different perspectives and in various settings.

Hitchin studied a limiting form of the self-dual equations of Yang-Mills theory in four dimensions,
imposing invariance under two of the dimensions.  The resulting equations were found to be
conformally invariant with respect to the remaining two directions, so had a natural formulation
as equations on a Riemann surface.  The Hitchin moduli space, or moduli space of Higgs bundles,
will be defined as the space of solutions modulo gauge transformations.

Several basic properties of the space of solutions emerged.  First, the equations could be
interpreted formally as hyperkahler moment map equations for the compact gauge Group action
on the infinite-dimensional space of connections.  Since we quotient solutions by gauge
transformations, this means we have an interpretation of the Hithchin moduli space as a hyperkahler
quotient.  Thus it should inherit a hyperkaher structure.  Of the three independent complex structures,
the one most amenable to algebraic geometry is the ``Dolbeault'' one inherited from the complex
structue on the Riemann surface, $C$.  In that setting, the Hitchin moduli space has an
interpretation as stable Higgs pairs, i.e. pairs $(V,\varphi)$, where $V$ is a rank-$n$ holomorphic
vector bundle over $C$, and $\varphi\in \Gamma(End(V)\otimes K_C)$ is a holomorphic $n\times n$
matrix-valued one-form.  Stability requires that all $\varphi$-invariant sub-bundles have lesser slope
than $V$.  Finally, solutions are taken modulo holomorphic isomorphism of vector bundles.

Another structure which emerges in this complex structure
is that of an algebraically completely integrable system.
Recall first that a complete integrable system on a symplectic manifold (``phase space'') is a
maximal set of Poisson-commuting proper functions (``Hamiltonians").  Specifying the values
of these conserved quantities gives a common level set which is a Lagrangian torus.
This gives another definition:  a fibration by Lagrangian tori to a base manifold where the
Hamiltonians take their values.  The fiber map for the Hitchin space is defined by all the
invariant polynomials $Tr\varphi^k$.

The other complex structures are called ``de Rham,'' in which the moduli space is of flat
$GL_n(\mathbb C)$ connections modulo complex gauge transformations, and the ``Betti'' complex
structure in which the moduli space is irreducible representations of $\pi_1(X)$ modulo conjugation.
A variety of theorems relate these.  Betti and de Rham are equated by the Riemann-Hilbert
correspondence, in which one assigns to a flat connection the monodromy data around loops in the space.

\section{Local Case}

Consider that a holomorphic vector bundle has a local trivialization by a holomorphic frame,
with respect to which the Higgs field looks like 

\section{Teichm\"uller theory}

\section{Link to Knots}

A hidden agenda for this seminar was to begin to explore the spectral curve from the perspective
of symplectic geometry.  Given a Riemann surface with complex structure $J$ and K\"ahler metric $h$,
it cotangent bundle $T^*C$ has a tautological holomorphic one-form $\theta$ and
holomorphic symplectic structure $\Omega = d\theta$.  
Treating $C$ as a real manifold, $T^*C$ has a canonical symplectic form $Re(\Omega)$
for which the spectral curve is Lagrangian, and we would like to consider it as an
object of the Fukaya category of $T^*C.$




%
%Now $h$ induces a K\"ahler metric
%$\omega$ on $T^*C,$ and the triple $(\omega,Re\Omega,Im\Omega)$ define the
%hyper-K\"ahler structure.  A holomorphic curve $\Sigma \subset T^*C$ is calibrated
%by $\omega$ but is Lagrangian with respect to $Re\Omega.$  In this symplectic structure,
%we can consider $\Sigma$ as an object in the Fukaya category of $T^*C$.
%In fact is special Lagrangian
%with respect to the Calabi-Yau form $\omega - i Im \Omega.$

\section{Mirror Symmetry}

Recall that mirror symmetry is a kind of equivalence between a manifold $X$ and its mirror $Y$.
It goes deep, but in its simplest formulation, the Hodge diamonds of $X$ and $Y$ are related
by a mirror reflection.
The basic philosophy behind mirror symmetry is that mirror manifolds are dual torus fibrations,
according to the following reasoning.  A B-type D-brane (or B-brane) on a complex manifold is a coherent sheaf,
and we will only consider the case of the structure sheaf of
single point.  An A-type D-brane (or B-brane) in a symplectic
manifold is a special Lagrangian submanifold with a flat connection $\nabla$
--- and we shall focus on the line bundle case.

On a complex manifold $X$, the space of B-branes connected to a point brane is $X$ itself.
If there were a mirror manifold $Y$, the corresponding space of A-branes should again be $X,$
since the moduli spaces of mirror branes should correspond.  So
we should be able to find the mirror $X$ to $Y$ by identifying A-branes that correspond to point B-branes
and looking at their moduli.  By reversing the reasoning,
we also learn that the manifolds $X$ and $Y$ each are expressible
both as a space of A-branes.

If we fix an A-brane $(T,\nabla)$, the space of A-branes deformation equivalent
to (``connected to'') the brane $(T,\nabla)$ therefore has a forget map to the space
of special Lagrangians connected to $T$.  The fibers are tori $(S^1)^{b_1(T)}$ which encode
the monodromies, and it so happens
that by  linearizing
the special Lagrangian equations you can see that $b_1(T)$ is also the dimension of the space
of special Lagrangians at $T.$  So the D-brane moduli space is an integrable system, a torus
fibration over special Lagrangian moduli space (Lagrangianicity must be argued separately).
Now $Y$, like $X,$ should itself be some an A-brane (on $X$) moduli space, so has a torus
fibration structure, a map $Y\to B$.  What if we took $T$ to be a torus fiber?
Then the dimensions are right, since we want $b_1(T)$ to be equal to the complex dimension
of $X$ and $Y,$ equivalently the real dimension of $T$.  A fiber also has the property that it is disjoing from its deformations -- just like
its dual point B-branes.  If this is the case (and there are other arguments supporting this which
we ignore), then the D-brane moduli space of a torus fiber $T$ fibers over the same base $B$ (its geometric
deformations), with the fiber over $T_b = \pi^{-1}(b)$ being the flat line bundles on $T_b,$ i.e. the
dual torus.  Therefore

$$\emph{Mirror pairs are dual Lagrangian torus fibrations}$$

The geometric Langlands program is a connection between Hitchin moduli spaces of
Langlands dual groups $G$ and $G^\vee.$
Each gives rise to an integrable system via its spectral curve construction.
Hausel-Thaddeus found these spaces to be dual torus fibrations
and computed the (stringy) Hodge numbers of both sides, revealing them to be mirror.





\chapter{Overview and Definitions}


\chapter{Symplectic Quotients in Finite and Infinite Dimensions}


\chapter{Hyperkahler Quotients and Integrable Systems}

%!TEX root = HitchinSystems.tex
\newcommand{\sA}{\mathcal{A}}
\newcommand{\sB}{\mathcal{B}}
\newcommand{\sE}{\mathcal{E}}
\newcommand{\sF}{\mathcal{F}}
\newcommand{\sG}{\mathcal{G}}
\newcommand{\sL}{\mathcal{L}}
\newcommand{\sO}{\mathcal{O}}
\newcommand{\sP}{\mathcal{P}}

\newcommand{\bZ}{\mathbb{Z}}
\newtheorem{lem}{Lemma}

\chapter{Stability and the Narasimhan-Seshadri Theorem}

In this chapter we will give an idea of how the classical theorem of
Narasimhan-Seshadri is proved. We follow mainly \cite{NS}. It states
the following:
\begin{thm}
Let $X$ be a compact Riemann surface, of genus $\geq 2$ then the
isomorphism classes\footnote{This means: representations are considered up to
  conjugation, algebraic vector bundles up to isomorphism of vector
  bundles and connections up to gauge fixing the imposed equation.} of
the following three sets are in bijection:
\begin{enumerate}[(i)]
\item $n$-dimensional irreducible unitary representations of
  $\pi_1(X)$;
\item stable rank $n$, degree $0$, algebraic vector bundles over $X$;
\item indecomposable rank $n$ holomorphic vector bundles with an
  unitary connection $A$, s.t. $F_A=0$\footnote{Here we denote by
    $F_A$ the curvature of the connection $A$.}.
\end{enumerate}
\end{thm}

\begin{rem*}
Actually Narasimhan-Seshadri result is more general. They consider a cover $p: Y
\rightarrow X$ ramified of degree $d \in [-n+1,0]$ at a single point $x_0 \in
X$, s.t. $Y/H \simeq X$ and $Y$ is simply connected. One modifies the
above classes as follows:
\begin{enumerate}[(i)']
\item homomorphisms $\rho: H \rightarrow U(n)$ of type $\xi_{-d}$,
  i.e. $\left.\rho\right|_{\mbox{Stab}_{x_0}}=\zeta_{-d}$\footnote{The
  stabilizer of any pre-image of $x_0$, which is cyclic of order $n$,
  acts by multiplication by a root of unity.};
\item stable rank $n$, degree $d$, algebraic vector bundles over $X$;
\item indecomposable rank $n$ holomorphic vector bundles over $X$, with an
  unitary connection $A$, s.t. $F_A=-2\pi i\frac{d}{n}$.
\end{enumerate}
\end{rem*}

\begin{rem*}
Stricly speaking Narasimhan-Seshadri theorem is only the equivalence
between (i)' and (ii)' above. A newer proof by Donaldson
\cite{D} stablishes the relation between (ii)' and
(iii)'\footnote{The relation between (iii)' and (i)' is an easy case
  of Riemann-Hilbert correspondence. For the degree $0$ case this can
  be seem directly.}. More importantly, their proof proceeds by
induction on the rank and they stablish the following result as well,
which is essential to make the inductive step work. Let $T$ be an
algebraic variety and $\left\{ E_t\right\}_{T}$ be a flat family of
algebraic vector bundles over $X$. Then $T_s \subset T$, the set of
points $t$, s.t. $E_t$ is stable, is Zariski open. Similarly for
semistable.
\end{rem*} 

\begin{rem*}
The relevance of this result to this class is that Hitchin proves a
more general result \cite[Proposition 3.4]{H1} by yet a different
method. Taking $\Phi=0$ recovers the case $n=2$ above. He remarks in
the introduction that one should be able to obtain the result for
arbitrary $n$, by considering $GL_n$. (I don't know of a reference for
this.)
\end{rem*}

From now one we will always be working in the algebraic category, that
is by vector bundle we will mean an algebraic vector bundle, and
families are algebraic families and so forth. We will also restrict to
the simpler form we first stated and give the elements of the proof for
it, the more general case is not much harder.

\section{Weil's theorem}

Firstly, we need to say a little bit about how to construct a vector
bundle from a representation of $\pi_1(X)$, and which ones come from
these. This is essentially the following result of Weil.

\begin{prop} 
An indecomposable vector $E$ over $X$ comes from a representation of
$\pi_1(X)$ if and only if $c_1(E)=0$.
\end{prop}

\begin{proof}
One direction is a construction. Let $p: Y \rightarrow X$ be a simply connected covering
of $X$, and $\rho: \pi_1(X) \rightarrow GL(V)$ a representation, then define 
\[\Phi(\rho) \equiv p_{*}\left(Y\times_{\pi_1(X)}V\right)^{\pi_1(X)},\]
where we confund $Y\times_{\pi_1(X)}V$ the associated vector bundle over $Y$
with the corresponding locally constant sheaf\footnote{The superscript $\pi_1(X)$
means we consider the invariant part with respect to the $\pi_1(X)$ action by
pullback.}. If $\dim(V) = n$ then $\Phi(\rho)$ is locally constant of
rank $n$.

For the converse, suppose $E$ is an indecomposable vector bundle
over $X$. Let $P$ denote the associated principal $GL_n$-bundle, then
one has the following exact sequence:
\[0 \rightarrow \mbox{ad}(P) \rightarrow TP^{GL_n} \rightarrow TX \rightarrow 0,\]
whose splitting is the data of an algebraic connection on $P$\footnote{The
middle term is the invariant vectors of $TP$ with respect to the
action of $GL_n$ induced from the $GL_n$-bundle structure, and
$\mbox{ad}(P)$ is the associated vector bundle to $P$ obtained by
taking the adjoint representation of $GL_n$ on its Lie algebra.}. There
exists an element $\mbox{at}(P) \in H^1(X,\mbox{Hom}(TX,\mbox{ad}(P)))$ which
measues when $P$ has a connection\footnote{This is the so-called
  Atiyah class of a principal $G$-bundle and the preceeding exact
  sequence defines the Atiyah (Lie) algebroid.}.
Then one has the following non-trivial fact \cite[Section 5]{A}:
\begin{lem}
If $E$ is indecomposable then $\mbox{at}(E) = c_1(E)$.
\end{lem}
So by assumption our $E$ admits an algebraic connection
$\nabla: E \rightarrow E\otimes\Omega^1_X$. Moreover since $X$ has complex
dimension 1, $\nabla$ is integrable. So $\ker \nabla$ is a locally
constant sheaf of vector spaces over $X$, i.e. a representation of
$\pi_1(X)$.
\end{proof}

A not so hard fact to check is that $\Phi$ is actually a
functor\footnote{From the category of finite dimensional
  representations of $\pi_1(X)$ to that of vector bundles over $X$.}, that is
\[\Phi\left(\rho,\rho'\right) = \mathcal{H}\mbox{om}_X(\Phi(\rho),\Phi(\rho')).\]

The important result now is:
\begin{lem} 
When restricted to unitary representations $\Phi$ is fully faithful,
i.e.
\[\mbox{Hom}(\rho,\rho') \simeq \mathcal{H}\mbox{om}_X(\Phi(\rho),\Phi(\rho')).\]
\end{lem}

\begin{proof}
Since the category of finite dimensional representations is
semisimple, it is enough to consider irreducible objects, so one can
suppose $(\rho,V)$ is trivial, then the statement becomes.
\[V'^{\pi_1(X)} \simeq H^0(\pi_1(X),V') \simeq
H^0(X,\Phi(V')),\]
for any unitary finite dimensional represetation $V'$.
Given $v \in V'^{\pi_1(X)}$ one has a section $s(y) = (y,v)$ of
$Y\times_HV$, which is $\pi_1(X)$-equivariant so it gives a section of
$\Phi(\rho')$. This clearly is zero only if $v$ is zero so that is
injective. 
Now for the surjectivity suppose one has a section $f \in
H^0(X,\Phi(\rho'))$, this is a function $\tilde{f}: Y \rightarrow V$, s.t.
\[\tilde{f}\circ h = \rho'(h) \tilde{f}.\]
Since $V$ is unitary, let $||\tilde{f}||^2 =
\left<\tilde{f},\tilde{f}\right>_{V}$. By definition
$||\tilde{f}||^2:Y \rightarrow \mathbb{C}$, with $||\tilde{f}||^2\circ h =
||\tilde{f}||^2$, so one gets a globally defined function on
$X$. Since $X$ is projective this function is constant. Now
$||\tilde{f}||^2$ constant implies that $\tilde{f}$ is also constant,
so $\rho'$ acts trivially, or $\tilde{f}$ factor through
$V'^{\pi_1(X)} \rightarrow V$.
\end{proof}

\section{Proof of main theorem}

As we mentioned before the proof proceeds by induction on two assetion
let's recall them for convinience.

\begin{thm}
Let $X$ be a compact Riemann surface of genus $\geq 2$.
\begin{enumerate}[(i)]
\item A rank $n$ vector bundle $E$ over $X$ of degree 0 is stable if and only if it
is given by $\Phi(\rho)$ for some homomorphism $\rho: \pi_1(X) \rightarrow
U(n)$, i.e. an unitary representation of $\pi_1(X)$.
\item For any family $T$ of vector bundles over $X$, the subset $T_s
  \subset T$ (resp. $T_{ss}$) of points whose fiber is a stable
  (resp. semistable) vector bundle is Zariski open.
\end{enumerate}
\end{thm}

We will essentially just prove (i). To prove (ii) one needs to
consider specific models for the moduli spaces considered above and
analise what the stability (or semistability) condition is inside
them. This is standard depending on how one construct these moduli
spaces.

\begin{proof}
\textit{Step 1 (line bundles):} The statement (ii) is trivial as stability is a
vacuuos condition for line bundles.
For (i) one recalls another definition of $c_1$ for line
bundles. Consider the exponential exact sequence\footnote{Strictly
  speaking, we use GAGA to be able to talk about the exponential map.}
of sheaves
\[0 \rightarrow \underline{\bZ} \rightarrow \sO_X \rightarrow \sO^{\times}_X 
\rightarrow 0.\]
Then one has a long exact sequence in cohomology
\[\cdots \rightarrow H^1(X;\bZ) \rightarrow H^1(X,\sO_X) \rightarrow H^1(X,\sO^{\times}_X)
\overset{c_1}{\rightarrow} H^2(X;\bZ) \rightarrow \cdots.\]
This gives that $c^{-1}_1(0) \simeq H^1(X;U(1)) \simeq
\mbox{Hom}(H_1(X;\bZ),U(1)) \simeq \mbox{Hom}(\pi_1(X),U(1))$\footnote{Here
  $H^k(X;-)$ denotes singular cohomology and $H^k(X,-)$ denotes
  coherent cohomology.}.

\textit{Step 2 (irreducible unitary representations exist):} Since
$\pi_1(X)$ could have only trivial unitary representations one needs some
calculation to show that the space of such is non-empty. For example
for $g=2$ (genus of X), one has generators $A_1,A_2,B_1,B_2$ for
$\pi_1(X)$ and one can take $B_1 =
\mbox{diag}(\xi_N,\xi^2_N,\ldots,\xi^N_N)$, where $\xi_N$ is the $N$th
root of unity, $A_2$ the matrix with $1$'s only in the diagonal above
the main diagonal, and $A_1=B^{-1}_1$ and $B_2 = A^{-1}_2$. A similar
construction works for $g > 2$ as well.

\textit{Step 3 (characterize the image of unitary and irreducible unitary
representations):} Let $\rho$ be an unitary representation, then
$\Phi(\rho)$ is semistable. Indeed, suppose there is $F \subset
\Phi(\rho)$, s.t. $\deg(F) \geq 0$, we will prove that $\deg(F) =
0$. By contradiction, let $\deg(F) > 0$, by taking the exterior power
$\Lambda^{\mbox{rk}(F)}F \subset \Lambda^{\mbox{rk(F)}}\Phi(\rho)
\equiv \Phi(\rho')$ one can suppose that $F$ has rank 1. Now let
$\rho' = \oplus \sigma_i$ with $\sigma_i$ irreducible, we have
projections:
\[p_i: \Phi(\rho') \rightarrow \Phi(\sigma_i).\]
By tensoring with degree $0$ line bundle $\sL$ one can suppose $F =
\sO_X(dx_0)$ for some $x_o \in X$ and $d>0$, notice that for the
representation $\rho'$ this just means tensoring with a character
$\xi_{\sL}$. Now if $\sigma_0$ is the trivial representation than
$\left.p_0\right|_F$ is zero, because $\mbox{deg}(\Phi(\sigma_0))$ and a
non-zero map between line bundles can not decrease the
degree\footnote{One sees this easily by taking the definition of
  degree as the number of zeros minus the number of poles of a rational section.}. Now
$H^0(X,F)\neq 0$, because $F=\sO_X(dx_0)$, however
$H^0(X,\Phi(\sigma_i)) = 0$ for any non-trivial irreducible
$\sigma_i$, by $H^0(X,\Phi(\sigma_i)) \simeq \sigma^{\pi_1(X)}_i = 0$
since $\sigma_i$ is 1-dimensional. This gives a contradiction with
non-zero $p_i$'s, since $F$ is a subbundle of $\Phi(\rho')$ it most be
trivial, which is a contradiction with $\mbox{deg}(F)>0$.
Notice that by using the induction hypothesis for smaller
representations and the result just proved for the semistable one
obtains that if $\rho$ is an irreducible unitary representation than
$\Phi(\rho)$ is stable.

\textit{Step 4 (the variety of representations):} One has a structure of
algebraic variety on the space of representations of a fixed dimension of
$\pi_1(X)$. Indeed these are given by a collection of matrices, $2g$ of
them, satisfying one algebraic equation. Now this gives an algebraic
space parametrizing vector bundles over $X$ obtained applying the
functor $\Phi$. Moreover, one can put also an algebraic structure on
the space of rank $n$ degree $d$ vector bundles over $X$. The set of indecomposable vector bundles inside these
forms an irreducible variety \footnote{Both of these statements follow
  form the representability of the Quot scheme, though \cite{NS} gives
  a more direct argument.}. By Weil's theorem this is equivalent to
the subvariety $S$ of $n$-dimensional representation whose associated
vector bundle is stable.

\textit{Step 5 (surjectivity):} Let $S$ be the subset of $n$-dimensional
representations such that $\Phi(S)$ is stable. Then $S$ is a Zariski
open set (assuming (ii)) of an irreducible variety (Step 4), hence
connected. Now let $U_0 = U\cap S$ where $U$ is the set of unitary
representations, and $U_0$ is the set of irreducible and unitary
representations (Step 3). So $U_0$ is open and closed inside of $S$,
which is connected, since $U_0$ is not empty (Step 2) $U_0=S$. This
finishes the proof.
\end{proof}



\chapter{Teichmueller Theory}

%% Matthew's talk #1
\chapter{Construction of the Hitchin Moduli Space}


This talk closely follows \cite{H1}. In this chapter, it is shown
that the Hitchin moduli space $\mathcal{M}_{H}$ is a smooth manifold
of dimension $12(g-1)$. We also prove that $\mathcal{M}_{H}$ is
equipped with a complete hyperkahler metric.


\section{Definitions}

Let $M$ be a Riemannian surface, $P\rightarrow M$ a principal $G=SO\left(3\right)$-bundle,
$V$ the associated rank 2 vector bundle, and $\mathcal{G}$ the group
of gauge transformations ($\mathcal{G}=Map\left(M,G\right)$). Recall
that for a connection $A$ on $P$ and $\Phi\in\Omega_{M}^{1,0}\left(adP\otimes\mathbb{C}\right)$,
the self-dual equations are
\begin{align*}
\tag{\ensuremath{\star}}F_{A}+\left[\Phi,\Phi^{*}\right] & =0,\\
\overline{\partial_{A}}\Phi & =0
\end{align*}


For a connection $A$, $u\in\mathcal{G}$ acts by
\[
u\left(A\right)=uAu^{-1}-\left(du\right)u^{-1}
\]
(where the covariant derivative is $\nabla_{A}=d+A\wedge$, i.e.,
\[
\nabla_{u\left(A\right)}s=u\nabla_{A}\left(u^{-1}s\right)
\]
for a section $s$ of $V$.

For $\Phi\in\Omega_{M}^{1,0}\left(adP\otimes\mathbb{C}\right)$, $u$
acts by
\[
u\left(\Phi\right)=u\Phi u^{-1},
\]
where we regard $adP\otimes\mathbb{C}$ as the bundle of trace-zero
endomorphisms of $V$.
\begin{defn}
The Hitchin moduli space is
\[
\mathcal{M}_{H}:=\left\{ \mbox{solutions to (}\star\mbox{)}\right\} /\mathcal{G}.
\]
\end{defn}
\begin{description}
\item [{Goal}] To learn about the geometry and topology of $\mathcal{M}_{H}$.
\end{description}

\section{Summary of results}

Let $V$ be a rank 2, odd degree vector bundle over a Riemannian surface
$M$. Then,
\begin{itemize}
\item $\mathcal{M}_{H}$ is a smooth manifold of dimension $12\left(g-1\right)$.
\item $\mathcal{M}_{H}$ has a natural metric.
\item $\mathcal{M}_{H}$'s metric is complete and hyperkahler (in fact,
$\mathcal{M}_{H}$ is a hyperkahler quotient).\end{itemize}
\begin{rem*}
$V$ has odd degree $\implies$ there are no reducible solutions to
$\left(\star\right)$ $\implies$ $\mathcal{G}$ acts freely on $\left(\star\right)$.
\end{rem*}

\section{$\mathcal{M}_{H}$ is a dimension $12\left(g-1\right)$ smooth manifold}

First, to get the expected dimension, we compute the dimension of
the tangent space to $\mathcal{M}_{H}$ at a regular point (i.e.,
one with trivial isotropy group).


\subsection{Idea of Proof}
\begin{itemize}
\item Linearize $(\star)$ to determine the expected dimension.
\item Let $\left(A\times\Omega\right)_{0}$ denote ``regular points,''
i.e., ones which are only fixed by the identity in $\mathcal{G}$
(those with trivial isotropy group). Then, exhibit $\mathcal{M}_{H}$
is a smooth submanifold of $\left(\mathcal{A}\times\Omega\right)_{0}/\mathcal{G}$
using the regular value theorem.
\end{itemize}

\subsection{Linearization of $(\star)$}

Let $\left(\dot{A},\dot{\Phi}\right)\in\Omega_{M}^{1}\left(adP\right)\oplus\Omega_{M}^{1,0}\left(adP\otimes\mathbb{C}\right)$.
To get the linearization of $(\star$), fix a base point $\left(A,\Phi\right)$
and look at
\[
\left.\frac{d}{dt}\right|_{t=0}\left(\star\right)\left(A+t\dot{A},\Phi+t\dot{\Phi}\right).
\]
Recalling that $F_{A}=dA+A\wedge A$, the first equation becomes
\[
d_{A}\dot{A}+\left[\dot{\Phi},\Phi^{*}\right]+\left[\Phi,\dot{\Phi}^{*}\right]=0,
\]
and the second is
\[
\overline{\partial_{A}}\dot{\Phi}+\dot{A}^{0,1}\wedge\Phi=0.
\]


$\left(\dot{A},\dot{\Phi}\right)$ arises from an infinitesimal gauge
transformation $\dot{\psi}\in\Omega_{M}^{0}\left(adP\right)$ if
\begin{align*}
\dot{A} & =d_{A}\dot{\psi}, & \dot{\Phi} & =\left[\Phi,\dot{\psi}\right].
\end{align*}
Let
\begin{eqnarray*}
d_{1}:\Omega_{M}^{0}\left(adP\right) & \longrightarrow & \Omega_{M}^{1}\left(adP\right)\oplus\Omega_{M}^{1,0}\left(adP\otimes\mathbb{C}\right)\\
\dot{\psi} & \mapsto & \left(d_{A}\dot{\psi},\left[\Phi,\dot{\psi}\right]\right)
\end{eqnarray*}
and
\begin{eqnarray*}
d_{2}:\Omega_{M}^{1}\left(adP\right)\oplus\Omega_{M}^{1,0}\left(adP\otimes\mathbb{C}\right) & \longrightarrow &
\Omega_{M}^{2}\left(adP\right)\oplus\Omega_{M}^{2}\left(adP\otimes\mathbb{C}\right)\\
\left(\dot{A},\dot{\Phi}\right) & \mapsto &
\left(d_{A}\dot{A}+\left[\dot{\Phi},\Phi^{*}\right]+\left[\Phi,\dot{\Phi}^{*}\right],\overline{\partial_{A}}\dot{\Phi}+\dot{A}^{0,1}\wedge\Phi\right).
\end{eqnarray*}
Then, $d_{1},d_{2}$ define an \emph{elliptic complex} with index
$3\left(2-2g\right)-6\left(g-1\right)=12\left(1-g\right)$. The Atiyah-Singer
index theorem then says that
\[
\dim H^{0}-\dim H^{1}+\dim H^{2}=12\left(1-g\right).
\]



\subsection{Elliptic Complexes}

Now, a short digression into elliptic complexes. Let $X$ be a manifold,
$\pi:T^{*}X\rightarrow X$ be projection onto the base, and $\left\{ E_{k}\rightarrow X\right\} $
a collection of vector bundles over $X$.
\begin{defn}
A chain complex
\[
\xymatrix{\cdots\ar[r] & \Gamma\left(E_{k-1}\right)\ar[r]^{P_{k-1}} & \Gamma\left(E_{k}\right)\ar[r]^{P_{k}} & \Gamma\left(E_{k+1}\right)\ar[r] & \cdots}
\]
is \textbf{elliptic} if the corresponding sequence of symbols
\[
\xymatrix{\cdots\ar[r] & \pi^{*}E_{k-1}\ar[r]^{\sigma\left(P_{k-1}\right)} & \pi^{*}E_{k}\ar[r]^{\sigma\left(P_{k}\right)} & \pi^{*}E_{k+1}\ar[r] & \cdots}
\]
is exact.\end{defn}
\begin{example*}
Let $P:E_{1}\rightarrow E_{2}$ be a differential operator. Then,
the complex
\[
\xymatrix{0\ar[r] & \Gamma\left(E_{1}\right)\ar[r]^{P} & \Gamma\left(E_{2}\right)\ar[r] & 0}
\]
is elliptic means that
\[
\xymatrix{0\ar[r] & \pi^{*}E_{1}\ar[r]^{\sigma\left(P\right)} & \pi^{*}E_{2}\ar[r] & 0}
\]
is exact, i.e., that $\sigma\left(P\right):\pi^{*}E_{1}\rightarrow\pi^{*}E_{2}$
is an isomorphism. Recall that this is the ``usual'' definition
of elliptic differential operator.
\end{example*}
Now, return to our elliptic complex.

\[
\xymatrix{\Omega_{M}^{0}\left(adP\right)\ar[r]^-{d_{1}} & \Omega_{M}^{1}\left(adP\right)\oplus\Omega_{M}^{1,0}\left(adP\otimes\mathbb{C}\right)\ar[r]^-{d_{2}} &
\Omega_{M}^{2}\left(adP\right)\oplus\Omega_{M}^{2}\left(adP\otimes\mathbb{C}\right)\\
\dot{\psi}\ar@{|->}[r] & \left(d_{A}\dot{\psi},\left[\Phi,\dot{\psi}\right]\right)\\
 & \left(\dot{A},\dot{\Phi}\right)\ar@{|->}[r] &
 \left(d_{A}\dot{A}+\left[\dot{\Phi},\Phi^{*}\right]+\left[\Phi,\dot{\Phi}^{*}\right],\overline{\partial_{A}}\dot{\Phi}+\dot{A}^{0,1}\wedge\Phi\right)
}
\]

By construction, we're interested in $H^{1}$ of this complex: $H^{0}=\ker d_{1}$
is the covariantly constant $\dot{\psi}$ which commute with $\Phi$.
These correspond to reducible solutions to $(\star$). So, if $H^{0}\neq0$,
then $\left(A,\Phi\right)$ is reducible. In fact, by considering
$d_{2}^{*}$, we can show that $H^{2}=0$ as well \cite{H1}. Hence,
\[
-\dim H^{1}=12\left(1-g\right),
\]
i.e., for a regular point $\left(A,\Phi\right)$,
\[
\boxed{\dim T_{\left(A,\Phi\right)}\mathcal{M}_{H}=12\left(g-1\right)}.
\]



\subsection{$\mathcal{M}_{H}$ is a smooth manifold}

This is a sketch, for complete details see \cite{H1}.
\begin{defn}
$\left(A,\Phi\right)$ is a \textbf{regular point} if the isotropy
group of $\left(A,\Phi\right)$ is the identity (i.e., there are no
nontrivial gauge transformations fixing this point).
\end{defn}
Recall that infinitesimally, these are the points where there are
no nonzero solutions to $d_{1}\dot{\psi}=0$.

Let $\left(\mathcal{A}\times\Omega\right)_{0}$ denote the open set
of regular points in $\mathcal{A}\times\Omega$, and
\[
B:=\left(\mathcal{A}\times\Omega\right)_{0}/\mathcal{G}.
\]
By construction, $B$ is a Banach manifold with the quotient topology.
We have the map
\[
d_{1}^{*}:\Omega_{M}^{1}\left(adP\right)\oplus\Omega_{M}^{1,0}\left(adP\otimes\mathbb{C}\right)\longrightarrow\Omega_{M}^{0}\left(adP\right).
\]
Define a \emph{slice} of $\mathcal{M}_{H}$ to be $\ker d_{1}^{*}$,
at some fixed $\left(A_{0},\Phi_{0}\right)$---then, the slices provide
coordinate patchs for $B$. Set
\begin{eqnarray*}
f:\ker d_{1}^{*} & \longrightarrow & \Omega_{M}^{2}\left(adP\right)\oplus\Omega_{M}^{2}\left(adP\otimes\mathbb{C}\right)\\
\left(A,\Phi\right) & \mapsto & \left(F_{A}+\left[\Phi,\Phi^{*}\right],\overline{\partial_{A}}\Phi\right);
\end{eqnarray*}
then, $f^{-1}\left(0,0\right)$ is a smooth submanifold of $\ker d_{1}^{*}$
with dimension $12(g-1)$.

Since $\ker d_{1}^{*}$ form coordinate patches for $B$, the remainder
of the proof is just arguing that the $\ker d_{1}^{*}$ patch together
to form a smooth manifold.


\section{The tangent space}

Thanks to our results in the previous section, we have an explicit
description of the tangent space to $\mathcal{M}_{H}$:
\[
T_{\left(A,\Phi\right)}\mathcal{M}_{H}=\left\{
\left(\dot{A},\dot{\Phi}\right)\left|\begin{matrix}d_{A}\dot{A}+\left[\dot{\Phi},\Phi^{*}\right]+\left[\Phi,\dot{\Phi}^{*}\right]=0,\\
\overline{\partial_{A}}\dot{\Phi}+\dot{A}^{0,1}\wedge\Phi=0,\\
d_{A}^{*}\dot{A}+Re\left[\Phi^{*},\dot{\Phi}\right]=0.
\end{matrix}\right.\right\}
\]
The third equation appears because $d_{1}^{*}\left(\dot{A},\dot{\Phi}\right)=d_{A}^{*}\dot{A}+Re\left[\Phi^{*},\dot{\Phi}\right]$.


\section{$\mathcal{M}_{H}$ has a complete hyperkahler metric}

Recall from Sean's talk that the metric on $\mathcal{A}\times\Omega$
given by
\[
g\left((\psi,\phi),\left(\psi,\phi\right)\right)=2i\int_{M}Tr\left(\psi^{*}\psi+\phi\phi^{*}\right)
\]
induces a metric on $T_{p}\mathcal{M}_{H}$. We want to show that
this is a complete metric on $T_{\left(A,\Phi\right)}\mathcal{M}_{H}$.
As a reminder:
\begin{defn*}
A metric on $M$ is \textbf{complete} if every Cauchy sequence of
points on $M$ has a limit which is also in $M$.
\end{defn*}

\subsection{Idea of Proof}

By contradiction: Suppose we have a sequence of points in $\mathcal{M}_{H}$
defined by a geodesic $\gamma$ converging to a point not in $\mathcal{M}_{H}$.
Because $g$ is $\mathcal{G}$-invariant, we can look at $\tilde{\mathcal{M}_{H}}\subset\mathcal{A}\times\Omega$.
Lift $\gamma$ to a horizontal $\tilde{\gamma}$ in $\tilde{\mathcal{M}}_{H}$.
$\tilde{\gamma}$ still defines a Cauchy sequence in $\tilde{\mathcal{M}}_{H}$,
so we have
\[
||A_{n}-\overline{A}||_{L^{2}}^{2}+||\Phi_{n}-\overline{\Phi}||_{L^{2}}^{2}\leq C
\]
 for some $C$ as $\left(A_{n},\Phi_{n}\right)\rightarrow\left(\overline{A},\overline{\Phi}\right)$
(where $\left(\overline{A},\overline{\Phi}\right)$ is the limiting
point not in $\mathcal{M}_{H}$).

Now, we apply Uhlenbeck's compactification theorem to show that there's
a gauge transformation taking $\left(\overline{A},\overline{\Phi}\right)$
to a solution of $(\star)$:
\begin{thm}
[Uhlenbeck] There are constants $\epsilon_{1}$, $M>0$ such that
any connection $A$ on the trivial bundle over $\overline{B}^{4}$
with $||F_{A}||_{L^{2}}<\epsilon_{1}$ is gauge equivalent to a connection
$\tilde{A}$ over $B^{4}$ with
\begin{enumerate}
\item $d^{*}\tilde{A}=0$,
\item $\lim_{\left|x\right|\rightarrow1}\tilde{A}_{r}=0$, and
\item $||\tilde{A}||_{L_{1}^{2}}\leq M||F_{\tilde{A}}||_{L^{2}}$.
\end{enumerate}
Moreover for suitable constants $\epsilon_{1}$, $M$, $\tilde{A}$
is uniquely determined by these properties, up to $\tilde{A}\mapsto u_{0}\tilde{A}u_{0}^{-1}$
for a constant $u_{0}$ in $U\left(n\right)$.
\end{thm}
In particular, we can use the following corrollary:
\begin{cor*}
For any sequence of ASD connections $A_{\alpha}$ over $\overline{B}^{4}$
with $||F\left(A_{\alpha}\right)||_{L^{2}}\leq\epsilon$, there is
a subsequence $\alpha'$ and gauge equivalent connections $\tilde{A}_{\alpha'}$
which converge in $C^{\infty}$ on the open ball.
\end{cor*}
Therefore, there's a gauge transformation taking $\left(\overline{A},\overline{\Phi}\right)$
to a solution of $(\star)$. This is a contradiction, so $\mathcal{M}_{H}$
is complete.


\subsection{Hyperkahler structure}

Recall from Sean's talk that there's a symplectic form on $\mathcal{M}_{H}$
given by
\[
\omega\left((\psi_{1},\phi_{1}),\left(\psi_{2},\phi_{2}\right)\right)=\int_{M}Tr\left(\phi_{2}\psi_{1}-\phi_{1}\psi_{2}\right).
\]
This defines a complex moment map from the action of $\mathcal{G}$:
\[
\mu\left(A,\Phi\right)=\overline{\partial}_{A}\Phi.
\]
We can write $\mu=\mu_{2}+i\mu_{3}$ and $\omega=\omega_{2}+i\omega_{3}$
to get two symplectic structures out of this. The third (or first?)
symplectic structure is just the Kahler form associated to the metric
\[
g=2i\int_{M}Tr\left(\psi^{*}\psi+\phi\phi^{*}\right),
\]
and has moment map
\[
\mu_{1}\left(A,\Phi\right)=F_{A}+\left[\Phi,\Phi^{*}\right].
\]
This exhibits $\mathcal{M}_{H}$ as a hyperkahler quotient of $\mathcal{A}\times\Omega$:
\[
\mathcal{M}_{H}=\bigcap_{i=1}^{3}\mu_{i}^{-1}\left(0\right)/\mathcal{G}.
\]



\section{Summary of topological results}

Here, we state some topological results. For proofs, see section 7
of \cite{H1}.

$\mathcal{M}_{H}$ is...
\begin{itemize}
\item non-compact
\item connected and simply connected
\item the Betti numbers $b_{i}$ vanish for $i>6g-6.$
\end{itemize}


\chapter{Integrable Systems and Spectral Curves}

\chapter{Meromorphic Connections and Stokes Data}
This talk is presented by Honghao on  April 22nd, 2015. The references is the work of P.~Boalch.

The talk has two parts. The first parts reviewed the three descriptions of the Hitchin moduli space: Dolbeault, De-Rham and Betti. The relation of the last two is also known as the Riemann-Hilbert correspondence. The correspondence can be generalized to punctured discs, and it requires additional information on each side. The additional packages on the two sides contain meromorphic connections and Stokes data.

\medskip
\textbf{Perspectives of the Hitchin space}

A definition first.

Let $X$ be a Riemann surface (probably not compact), and $G$ be a Lie group over $\mathbb{C}$. The character variety is defined to be
$$\mathcal {M} = Hom (\pi_1(X),G)/G.$$

The three descriptions of Hitchin's moduli space:
\begin{enumerate}
  \item (Dolbeault) $\mathcal{M}_{Dol}$ the moduli space of Higgs bundles, which consists of pairs $(E,\Phi)$, where $E$ is a rank $n$ degree zero holomorphic vector bundle and $\Phi\in \Gamma(End(E)\otimes \Omega^1)$ a Higgs field.
  \item (De-Rham) $\mathcal{M}_{DR}$ the moduli space of connections on rank $n$ holomorphic vector bundles, consisting of pairs $(V, \nabla)$ with $\nabla: V\rightarrow V \otimes \Omega^1$ a holomorphic connection.
  \item (Betti) $\mathcal{M}_{B}$ the conjugacy classes of representation of the fundamental group of $X$. Notice this is the character variety of the compact Riemann surface with $G=GL(n,\mathbb{C})$.
\end{enumerate}

From Dolbeault to De-Rham: naturally diffeomorphic as real manifolds via the non-abelian Hodge correspondence, but not complex analytically isomorphic.

From De-Rham to Betti: Locally, the connection can be written as $\nabla = d + A$. Since $X$ is compact, homomorphic is the same as algebraic and $V$ and $\nabla$ are holomorphic implies $A\in GL(n,\mathbb{C})$. The flatness of the connection implies the holonomy only depends on the homotopy type, thus the representation of the fundamental group $\pi_1(X)$.

Example: when $n=1$ that is $G = \mathbb{C}^*$, then
\begin{enumerate}
  \item $\mathcal{M}_{Dol} \cong T^*Jac(X)$.
  \item $\mathcal{M}_{DR} \rightarrow Jac(X)$ a twisted cotangent bundle of the Jacobian of $\Sigma$; it is an affine bundle modeled on the cotangent bundle.
  \item $\mathcal{M}_{B} \cong (\mathbb{C}^*)^{2g}$ is isomorphic to $2g$ copies of $\mathbb{C}^*$.
\end{enumerate}

Alert: The isomorphism (Riemann-Hilbert correspondence) $\mathcal{M}_{DR} \rightarrow \mathcal{M}_{B}$ involves exponential and not algebraic. And conversely, no algebraic isomorphism $\mathcal{M}_{B}\rightarrow \mathcal{M}_{DR}$ exists.

Another argument: $\mathcal{M}_{B}$ is affine, and it does not have compact subvarieties. Thus so is $\mathcal{M}_{DR}$. However, $\mathcal{M}_{Dol}$ has such (zero section), which make it impossible to have complex analytic isomorphism $\mathcal{M}_{DR} \rightarrow \mathcal{M}_{Dol}$. On the other hand, the abelian Hodge theory and Dolbeault isomorphism gives rise to a non-holomorphic isomorphism $\mathcal{M}_{DR} \rightarrow \mathcal{M}_{Dol}$.

A couple of remarks,

1) Dolbeault space has algebraic Hamiltonian integrable system (presented by Lei, also known as the Hitchin system?), there is a proper map
$$\mathcal{M}\rightarrow \mathbb{H}$$
to a vector space of half of the dimension. The generic fibres of the map are abelian varieties. In the abelian case, the space is product of $\mathbb{C}^g\times Jac(X)$, but in general, the fibres vary non-trivially and there are singular fibres.

2) The mapping class group has an natural (symplectic, algebraic) action on the moduli space, through the Betti description.


\medskip
\textbf{Airy's equation}

Consider $X = \mathbb{P}^1 = \mathbb{C}_z \cup \{\infty\} = \{0\}\cup \mathbb{C}_w$. The general Airy's equation: $f^{''} = z^n f$. Write $\nabla = d - A$, the ODE corresponds to a holomorphic connection on $\mathbb{C}_z$, as in
$$\frac{d}{dz}\begin{pmatrix}
      f \\
      f'
    \end{pmatrix}
  = \begin{pmatrix}
  0 & 1 \\ z^n & 0
  \end{pmatrix}\begin{pmatrix}
      f \\
      f'
    \end{pmatrix}.$$
However, this connection is singular at infinity. Let $w =1/z$, then $z\partial_z = - w\partial_w$. The ODE becomes $((z\partial_z)^2 - (z\partial_z) - z^{n+2})f = 0$, which in the new coordinate is $((w\partial_w)^2 + (w\partial_w) - w^{-n-2})f = 0$, that is
$$\frac{\partial^2f}{\partial w^2} + \frac{2}{w}\frac{\partial f}{\partial w} - \frac{1}{w^{n+4}}f =0.$$
In terms of connection, this is
$$\frac{d}{dw}\begin{pmatrix}
      f \\
      f'
    \end{pmatrix}
  = \begin{pmatrix}
  0 & 1 \\ \frac{1}{w^{n+4}} & \frac{2}{w}
  \end{pmatrix}\begin{pmatrix}
      f \\
      f'
    \end{pmatrix}.$$
The connection $A$ can be expanded as
$$A = \frac{A_{n+4}}{w^{n+4}} + \frac{A_1}{w} + \text{holomorphic terms},$$
where $A_i \in \mathfrak{gl}(n,\mathbb{C})$.

The irregular type $Q$ of this connection at infinity is
$$dQ = \frac{A_{n+4}}{w^{n+4}} = \frac{1}{w^{n+4}}\begin{pmatrix} 0 & 0 \\ 1 & 0\end{pmatrix}$$

\medskip
\textbf{Regular singularity}

A meromorphic connection has regular singularity if its singular pole has order ar most $1$. Deligne showed a version of the correspondence.

Deligne's Riemann-Hilbert correspondence: If $X$ is a Riemann-Surface with punctures and $G =GL(n,\mathbb{C})$, then the character variety corresponds to the space of algebraic connections with regular singularities at the punctures.

\medskip
\textbf{Stokes Data}

Let $D$ be an effective divisor on $\mathbb{P}^1$. A meromorphic connection $\nabla$ on $D$ is a map $\nabla: V\rightarrow V\otimes K(D)$ satisfying the Leibniz rule. Locally at an irregular singular point, $\nabla = d - A$ and $A = dQ + \Lambda \frac{dz}{z}$, $Q$ is a matrix of meromorphic functions. Choose a framing so that $Q$ is diagonal, that is $Q = \text{diag}(q_1,q_2,\dotsb, q_n)$. Let $q_{ij}(z)$ be the most singular term on $q_i - q_j$.

Some definitions.
\begin{itemize}
  \item Let $S^1$ parameterize the rays around $z$. If $d_1,d_2\in S^1$, define $Sect(d_1,d_2)$ be the open sector sweeping from $d_1$ to $d_2$.
  \item The anti-Stokes directions $\mathbb{A} \subset S^1$ are the directions $d\in S^1$ such that for some $i~\neq~j$,$q_{ij}~\in~\mathbb{R}_{<0}$ along this ray $d$.
  \item The roots of $d$ are the ordered pairs (ij) supporting $d$:
  $$Roots(d) := \{(ij)| q_{ij}\in \mathbb{R}_{<0} \text{ along } d\}.$$
  \item The multiplicity of $d$ is the number of roots supporting $d$.
  \item the group of Stokes factors associated to $d$ is
  $$Sto_d(A) := \{K\in G |(K)_{ij} = \delta_{ij}, \text{ unless }(ij) \text{ is a root of } d\}.$$
  This is a unipotent subgroup of $G = GL(n,\mathbb{C})$.
\end{itemize}

Remarks: The anti-Stokes directions are those where the roots decays rapidly towards the singular point.

To see the stokes group is unipotent. First, $i\neq j$ implies the diagonals are always $1$. Second, if $(ij)$ is a root, $(ji)$ won't be, which implies an ordering. Third, transitivity, as $q_{ij}, q_{jk}\in \mathbb{R}_{<0}$, then $q_{ik} = q_{ij} + q_{jk}\in \mathbb{R}_{<0}$. By a permutation, the Stokes matrix becomes upper triangular, with diagonals equal to $1$, which is obviously an unipotent subgroup, and the property of which is invariant under permutation conjugation.

Since $q_{ij}(z) = a/z^{k-1}$, there is a $\pi /(k-1)$ rotational symmetry. ($k$ is the order of the pole.) Notice that in the generic situation, the leading terms of $q_i$ do not cancel out. $q_{ij}$ has a $2\pi /(k-1)$ symmetry, and $q_{ji}$ contributes the remaining. Let $\textbf{d}=(d_1,\dotsb,d_l)$ be the set of anti-Stokes directions up to this symmetry, then $n(n-1)/2 = \sum_{i=1}^l Mult(d_i)$.

A key result.
\begin{enumerate}
  \item The product of the groups of Stokes factors in a half-period is isomorphic to a subgroup of G as a variety.
    $$\prod_{d\in \textbf{d}}Sto_d(A) \cong PU_+P^{-1},$$
    via $(K_1,\dotsb,K_l)\mapsto K_l \cdot \cdot \cdot K_2K_1$, and $P$ is a permutation group arranging the anti-Stokes directions in order.
  \item The product of all groups of Stokes factors is isomorphic to the variety:
    $$\prod_{d\in \mathbb{A}}Sto_d(A) \cong (U_+\times U_-)^{k-1}.$$
\end{enumerate}

Suppose $E_{ij}$ is the matrix with the $(ij)$ entry $1$ and $0$ otherwise. It suffices to show any upper triangular matrix $U$ can be decomposed uniquely into a product of $(I+t_{ij}E_{ij})$, for $1\leq i < j\leq n$. Notice that $E_{ij}E_{kl} = \delta_{jk}E_{il}$. Expanding the product, $t_{12}, t_{23},\dotsb$ the sub-diagonal entries are determined immediately. Followed by the next sub-diagonal and so on. The second fact follows the first one easily.

The second product is the space of Stokes data $Sto(Q)$. For $k=2$, the (local version) irregular Riemann-Hilbert correspondence is
$$\{\text{Connections of singular type Q}\}/\mathcal {G} \cong \mathfrak{t} \times Sto(Q),$$
where $\mathcal {G}$ is the group of holomorphic maps from the unit disk to $G$ taking $z=0$ to $1_G$, and $\mathfrak{t}$ is an Cartan subalgebra fixed a priori.


\chapter{The Langlands Program and Relations to Geometric Langlands}


\chapter{Spectral Curves and Irregular Singularities}


\chapter{The Stokes Groupoid}


\chapter{Cluster Varieties}


\chapter{The Hitchin System and Teichmueller Theory II}


\chapter{Geometric Langlands and Mirror Symmetry}

% Matthew Talk 2
\chapter{Hitchin Systems and Supersymmetric Field Theories}

\section{Introduction and definitions}

In previous talks, we've explored many properties of the Hitchin system
and its geometry, as well as several applications. The core idea of
this talk is that, given a certain supersymmetric field theory, we
can obtain the Hitchin system as a moduli space associated to this
theory via a standard procedure in quantum field theories. Gaiotto,
Moore, and Neitzke \cite{GMN} use this approach to construct a canonical
coordinate system on the Hitchin moduli space. The goal of this talk
is to explain some of the language of supersymmetric quantum field
theories, and describe how we can obtain $\mathcal{M}_{H}$ from a
particular class of such theories.


\subsection{Supersymmetry}

Consider $\mathbb{R}^{n}$. The \textbf{Poincare group} is the group
of translations and rotations of this space: $G=ISO(n)\cong\mathbb{R}^{n}\rtimes SO(n)$.
(In Minkowski signature, say for $\mathbb{R}^{n,1}$, we might write
$ISO(n,1)$ instead). Recall that there's a spin group defined as
a double cover of $SO(n)$:
\[
\xymatrix{1\ar[r] & \mathbb{Z}_{2}\ar[r] & Spin(n)\ar[r] & SO(n)\ar[r] & 1.}
\]
When we have such a double cover, we get an extension of the group
$G$ to the supergroup $\tilde{G}$. This gives an extension of the
Poincare algebra $\mathfrak{g}$ to the \textbf{super Poincare algebra}
$\tilde{\mathfrak{g}}$ coming from the double-cover of $SO(n)$.
$\tilde{\mathfrak{g}}$ is called the \textbf{supersymmetry algebra}.
Such $\tilde{g}$ are labelled by a choice of a representation of
$Spin(n)$. For irreducible representations of $Spin(n)$, there are
two possible cases:
\begin{itemize}
\item There is either a unique irreducible spinor representation $S$, and
any spinor representation has the form $S^{\oplus N}$ for some $N$,
or
\item There are two distinct irreducible real spinor representations $S_{+},S_{-}$,
and any spinor representation has the form $S_{+}^{\oplus N_{1}}\oplus S_{-}^{\oplus N_{2}}$
for some $N_{1},N_{2}$. (This occurs in dimensions $n\equiv2,6\mod8$).
\end{itemize}
Thus, when someone says $N=n$ or $N=\left(n_{1},n_{2}\right)$ supersymmetry,
they are specifying which extension of the Poincare algebra they're
referring to.

The supersymmetry Lie algebra $\tilde{\mathfrak{g}}$ splits into
an even and odd part:
\[
\tilde{\mathfrak{g}}=\mathfrak{g}^{0}\oplus\mathfrak{g}^{1},
\]
and is equipped with a skew-symmetric bracket $\left[\cdot,\cdot\right]:\tilde{\mathfrak{g}}\rightarrow\tilde{\mathfrak{g}}$
which satisfies the Jacobi identity. Note that $\left[\mathfrak{g}^{1},\mathfrak{g}^{1}\right]\subset\mathfrak{g}^{0}$,
$\mathfrak{g}^{0}=\mathfrak{g}$ (the Poincare algebra), and $\mathfrak{g}^{1}$
is just a linear representation of $\mathfrak{g}^{0}$.
\begin{example*}
For $N=2$ SUSY and $G=ISO(3,1)$, we'll be interested in a $\tilde{G}$
whose Lie algebra has even part
\[
\mathfrak{g}^{0}=\mbox{\ensuremath{\mathfrak{iso}}}(3,1)\oplus\mathbb{C}.
\]
The abelian $\mathbb{C}$ factor is central in $\tilde{\mathfrak{g}}$,
and has a canonical generator $Z$.
\end{example*}

\subsection{BPS States}

Let $\mathcal{H}$ denote the Hilbert space of states of our quantum
system on $\mathbb{R}^{3,1}$. We want to think of a state as being
labeled by a representation of $\tilde{G}$---the representation encodes
the data of the particle. For example, the vacuum state (``empty
space'') in $\mathcal{H}$ corresponds to the trivial representation
of $\tilde{G}$. The next simplest kind of state is one where space
is empty except for a single particle propogating with some definite
momentum $p\in T^{*}\mathbb{R}^{3,1}$. Call the subspace of the Hilbert
space consisting of single-particle states $\mathcal{H}^{1}$. Here,
there's some structure:
\begin{itemize}
\item $\mathcal{H}^{1}$ splits into components $\mathcal{H}_{M}^{1}$,
labeled by $M\in\mathbb{R}_{\geq0}$. $M^{2}$ is the eigenvalue of
a quadratic Casimir operator in $ISO(3,1)$. Physically, it's the
mass of the particle.
\item For $N=2$ SUSY as in our prior example, we also have a central generator
$Z\in\mathbb{C}$. Together, these give a decomposition
\[
\mathcal{H}^{1}=\bigoplus_{M,Z}\mathcal{H}_{M,Z}^{1}.
\]

\end{itemize}
Fix some momentum $p_{rest}\in\left(\mathbb{R}^{3,1}\right)^{*}$
with $||p_{rest}||^{2}=M^{2}$, and consider the subspace $\mathcal{H}_{M,Z}^{1,rest}$
on which the subgroup of translations along $\mathbb{R}^{3,1}$ acts
by $p_{rest}$. This is a representation of a subgroup $\tilde{G}_{rest}\subset\tilde{G}$
with
\[
\tilde{G}_{rest}=SO\left(3\right)\ltimes\tilde{T},
\]
where the ``super translation group'' $\tilde{T}$ is generated
by the ordinary translations $T=\mathbb{R}^{3,1}$ plus the central
character $Z$ and the ``odd translations'' $\mathfrak{g}^{1}$.
The odd translations act by a Clifford algebra (on an 8-dimensional
vector space). We can then count the number of unitary irreducible
representations of this Clifford algebra:
\begin{itemize}
\item If $M<\left|Z\right|$, then there are \emph{no} unitary representations
of the Clifford algebra.
\item If $M=\left|Z\right|$, the Clifford algebra is degenerate, and its
unique unitary irrep $S$ has dimension $2^{4/2}=4$.
\item If $M>\left|Z\right|$, the Clifford algebra is nondegenerate, and
its unique unitary irrep $S$ has dimension $2^{8/2}=16$.
\end{itemize}
States that satisfy $M=\left|Z\right|$ are called \textbf{BPS} (Bogomol'nyi,
Prasad, Sommerfield) \textbf{states}, and as one might expect, they
satisfy a set of differential equations depending on the field theory.
The BPS states are those in which half of the supersymmetry generators
are unbroken.


\subsection{Moduli of Vacua}

The ``moduli space'' associated to a QFT typically refers to the
moduli space of vacua. By \textbf{vacua}, we mean the quantum state
with the lowest possible energy.
\begin{description}
\item [{Question}] In what sense is the space of vacua a moduli space?
\end{description}
For scalar fields, these are labelled by the \textbf{vacuum expectation
value} (VEV). The VEV of an operator is (as the name suggests) the
expectation value of the operator in the vacuum (the quantum state
with the lowest possible energy). We can label a vacuum state by its
VEV, and this gives a moduli space of vacua.

For $N=2$ SUSY, the superalgebra has two representations with scalars:
\textbf{vectormultiplets} (one complex scalar), and \textbf{hypermultiplets}
(two complex scalars). This gives a local splitting of the moduli
of vacua $\mathcal{M}$ as
\[
\mathcal{M}=\mathcal{M}_{C}\oplus\mathcal{M}_{H},
\]
where $\mathcal{M}_{C}$ is the ``Coulomb branch'' (vectormultiplets),
and $\mathcal{M}_{H}$ is the ``Higgs branch'' (hypermultiplets).


\section{Compactification and Dimensional Reduction}

Compactification of a field theory is a process where, instead of
considering a general space $X$, we consider $X=M\times C$ where
$C$ is some compact space. Dimensional reduction is the limit of
the compactified theory where the volume of the compact space is shrunk
to zero, which produces an effective theory on the remaining dimensions.
\begin{example*}
[Toy Example] Consider a field theory on $X=\mathbb{R}^{n}\times S_{R}^{1}$
($S_{R}^{1}$ is the circle of radius $R$). Let $\theta$ be a coordinate
on $S_{R}^{1}$, and $x^{i}$ coordinates on $\mathbb{R}^{n}$. At
a fixed $x$ coordinate, the fields along the $S_{R}^{1}$ look like
\[
\phi|_{x}=\sum_{n}A_{n}\cos\left(\frac{2\pi n}{R}\theta\right)+B_{n}\sin\left(\frac{2\pi n}{R}\theta\right),
\]
where the coefficients $A_{n}$ and $B_{n}$ are determined by the
boundary conditions on $\phi$. As $R\rightarrow0$, the eigenvalues
$\lambda_{n}=\frac{2\pi n}{R}$ approach $\infty$, except for $n=0$.
Note that in quantum mechanics, $\hbar\lambda_{n}$ is the \emph{momentum}
of eigenstate $n$, so as $R\rightarrow\infty$, all mometums except
the trivial one also $\rightarrow\infty$. We should interpert $R\rightarrow0$
as meaning that, for finite energy (and hence, finite momentum), the
only eigenstate left is the trivial one.

If $\phi|_{x}$ is constant, it means that the field $\phi$ does
not depend on $\theta$---the dimensional reduction of the theory
on $S_{R}^{1}$ consists of the fields of the $\mathbb{R}^{n}\times S_{R}^{1}$
theory which do not depend on $\theta$.
\end{example*}
So, we have two equivalent perspectives on dimensional reduction to
$M$ of a theory on a space $M\times C$:
\begin{itemize}
\item It's the limit of the theory on $M\times C$ where the volume of $C$
contracts to zero, or
\item It's the theory on $M\times C$ where all fields are taken to be independent
of coordinates on $C$.
\end{itemize}

\begin{example*}
[Yang-Mills] We actually already encountered dimensional reduction
in one of the first lectures of the course. Consider classical Yang-Mills
theory.

Yang-Mills theory is a field theory defined for principal $G$-bundles
$P\rightarrow X$, where $X$ is a 4-dimensional Riemannian manifold.
\begin{description}
\item [{Fields}] Connections $A$ on $P$.
\item [{Lagrangian}]
\[
L\left(A\right)=\left|F_{A}\right|^{2}d\mu
\]

\end{description}
Recall that from the Lagrangian, we obtain the action functional by
\[
S\left(A\right)=\int_{X}L\left(A\right)=\int_{X}\left|F_{A}\right|^{2}d\mu.
\]
The equations of motion for Yang-Mills theory are
\[
d_{A}^{*}F_{A}=0,
\]
and the instantons are the (anti) self-dual connections:
\[
F_{A}=\pm*F_{A}.
\]
(Here, $*:\Omega_{X}^{2}\cong\Omega_{X}^{2}$ is the Hodge star operator).

In local coordinates we can write $d_{A}=d+A$, where
\[
A=A_{1}dx_{1}+A_{2}dx_{2}+A_{3}dx_{3}+A_{4}dx_{4}.
\]
Define
\[
F_{ij}:=\left[\nabla_{i},\nabla_{j}\right]=\frac{\partial}{\partial x_{i}}A_{j}-\frac{\partial}{\partial x_{j}}A_{i}+\left[A_{i},A_{j}\right].
\]
Look at the self-dual connections. Then, the instanton equation $F_{A}^{-}=0$
becomes
\begin{eqnarray*}
F_{12} & = & F_{34},\\
F_{13} & = & -F_{24},\\
F_{14} & = & F_{23}.
\end{eqnarray*}


Let's restrict attention to $X=C\times\mathbb{R}^{2}$, where $C$
is a Riemann surface. Suppose that we compactify and perform dimensional
reduction in $\mathbb{R}^{2}$ coordinates: restrict to $A_{j}$ which
are invariant under translation in $x_{3}$, $x_{4}$. Then, $A_{1}dx_{1}+A_{2}dx_{2}$
defines a connection on $C$. Relabel $A_{3}=\phi_{1}$ and $A_{4}=\phi_{2}$,
and define $\varphi=\phi_{1}-i\phi_{2}$; then the self-dual equations
become
\begin{eqnarray*}
F_{A}-\frac{1}{2}i\left[\varphi,\varphi^{*}\right] & = & 0,\\
\left[\nabla_{1}+i\nabla_{2},\varphi\right] & = & 0.
\end{eqnarray*}
If we think of $\varphi$ as defining a local section of $\Omega^{0}\left(C;ad(P)\otimes\mathbb{C}\right)$,
and set $\Phi=\frac{1}{2}\varphi dz\in\Omega^{1,0}\left(ad\left(P\right)\otimes\mathbb{C}\right)$
and $\Phi^{*}=\frac{1}{2}\varphi^{*}d\overline{z}\in\Omega^{0,1}\left(ad(P)\otimes\mathbb{C}\right)$,
then the equations become
\begin{eqnarray*}
F_{A}+\left[\Phi,\Phi^{*}\right] & = & 0,\\
\overline{\partial_{A}}\Phi & = & 0,
\end{eqnarray*}
the usual Hitchin equations.
\end{example*}

\section{5D Super Yang-Mills Theory}

Now, let's repeat the previous example, but including some supersymmetry.
5D Super Yang-Mills theory admits a conventional Lagrangian description:
Let $P$ be a principal $G$-bundle over $X$ (a 5-dimensional space).
The theory has:
\begin{description}
\item [{Fields}] Connections $A$ on $P$, sections $\phi^{i}$ of $Ad\left(P\right)$
($i=1,\ldots,5$), and fermions.
\item [{Lagrangian}]
\[
L=\frac{R}{8\pi^{2}}Tr\left[\frac{1}{R^{2}}F_{A}\wedge*F_{A}+\sum_{i=1}^{5}d_{A}\phi^{i}\wedge*d_{A}\phi^{i}+\mbox{fermions}\right].
\]
\end{description}
\begin{rem*}
~
\begin{enumerate}
\item 5D SYM is well-defined as an effective field theory, below a certain
energy scale. It is not obviously well-defined at arbitrarily high
energies.
\item There's this unusual $R$ factor appearing here that you should be
suspicious of. We'll explain where this comes from at the end of the
talk.
\end{enumerate}
\end{rem*}

\subsection*{Compactification on $C$}

Let's take $X=\mathbb{R}^{2,1}\times C$, where $C$ is a Riemann
surface. Analogous to the classical case, when we compactify 5D SYM
on $C$, we combine $\phi^{4}$ and $\phi^{5}$ into a complex-valued
1-form on $C$:
\[
\varphi=\left(\phi^{4}+i\phi^{5}\right)dz.
\]
Note that to be a sensible theory, we additionally require translation 
invariance along $\mathbb{R}^{2,1}$.
\begin{description}
\item [{Question}] What are the classical field configurations in the compactified
theory which preserve the supersymmetry? (Recall that these are the
BPS states!)
\end{description}
Assuming $\phi^{1},\phi^{2},\phi^{3}=0$, the equations satisfied
by the remaining fields are
\[
\tag{\ensuremath{\star}}\begin{cases}
 & F_{A}+R^{2}\left[\varphi,\varphi^{*}\right]=0,\\
 & \overline{\partial_{A}}\varphi=0,
\end{cases}
\]
which we recognize as (almost) Hitchin's equations. In other words,
the moduli space of vacua of $SYM[C]$ in the low energy limit is
\[
M_{C}\left[G\right]=\left\{ \mbox{solutions to }(\star)\right\} /\left\{ \mbox{gauge transformations}\right\} =\mathcal{M}_{H}.
\]

\begin{rem*}
We took $\phi^{1}=\phi^{2}=\phi^{3}=0$ above. If we don't, SUSY also
imposes equations on $\phi^{1},\phi^{2},\phi^{3}$:
\begin{align*}
d_{A}\phi^{i} & =0, & \left[\varphi,\phi^{i}\right] & =0, & \left[\phi^{i},\phi^{j}\right] & =0.
\end{align*}
But, at a generic point in the moduli space, these equations won't
have any nontrivial solutions, so the assumption that $\phi^{j}=0$
isn't much of an imposition.
\end{rem*}

A key difference between this example and dimensional reduction for classical
Yang-Mills theory is that we have dimensionally reduced to a theory on $\mathbb{R}^{2,1}$,
\emph{not} a theory on $C$. Instead of seeing Hitchin's moduli space as the moduli of
instantons for our theory, it appears as the moduli of BPS states!

The full moduli space of vacua has a Coulomb branch---identified with
the Hitchin moduli space---and Higgs branches attached to the specific
other points where nontrivial solutions for the $\phi^{j}$ exist.
(Unfortunate nomenclature: the moduli of Higgs bundles is the space
of solutions that live on the Coulomb branch...)


\section{Compactification from (2,0) 6D Theory}

Now let's talk about where that pesky $R$ factor came from. It turns
out that there's a famous 6D $N=\left(2,0\right)$ QFT. It doesn't
have a conventional Lagrangian description (or even a space of fields).
Instead, the inputs are a 6-dimensional manifold, together with a
Lie algebra $\mathfrak{g}$. Call this theory $X_{\mathfrak{g}}$.
It has the following properties:
\begin{itemize}
\item $X_{\mathfrak{g}}$ has $N=\left(2,0\right)$ SUSY in $d=6$.
\item $X_{\mathfrak{g}}$ has no parameters---no coupling constants or scale,
and the strength of the interaction can't be perturbed.
\item $X_{\mathfrak{g}}$ is conformally invariant.
\end{itemize}
Despite its unconventional description, we can still compactify $X_{\mathfrak{g}}$
to obtain lower-dimensional theories. In fact, 5D SYM is $X_{\mathfrak{g}}\left[S^{1}\right]$,
where the $R$ is the length of the $S^{1}$. So, $\mathcal{M}_{H}$
is obtained as the moduli space associated to $X_{\mathfrak{g}}\left[C\times S^{1}\right]$.
We could perform this compactification in either order: $\mathcal{M}_{H}$
can also be obtained as the moduli space associated to the theory
$X_{\mathfrak{g}}\left[C\right]$ compactified on $S^{1}$. \cite{GMN}
use this observation to produce canonical Darboux coordinate systems
on $\mathcal{M}_{H}$ and construct Calabi-Yau metrics in these coordinate
systems.

Some examples of information we can obtain from this perspective:
\begin{itemize}
\item Compactify $X_{\mathfrak{g}}$ on $C$ first to get a 4d $N=2$ supersymmetric
gauge theory with with Coulomb branch $\mathcal{B}$. Then, $\mathcal{B}$
is actually the Hitchin base, i.e., $\mathcal{M}_{H}\rightarrow\mathcal{B}$
with generic fiber a torus. Points $u\in\mathcal{B}$ correspond to
spectral curves $\Sigma_{u}\subset T^{*}C$, also known as ``Seiberg-Witten
curves.''
\item $\mathcal{M}_{H}$ is automatically hyperkahler because of supersymmetry.
\end{itemize}

\chapter{Pyongwon's II}

\chapter{Phil's II}

\chapter{Lei's II}

\chapter{Honghao's II}

%% To add another chapter, create a new .tex file (e.g. 21-Peng-III.tex) and 
%% add the corresponding command: \input{21-Peng-III.tex}


%% Bibliography
\begin{thebibliography}{AB}
\bibitem[A]{A}M.~F.~Atiyah, ``Complex analytic connections in fibre
bundles.'' Transactions of the American Mathematical Society (1957):
181-207.

\bibitem[AB]{AB}M.~F.~Atiyah and R.~Bott, The Yang-Mills equations
over Riemann surfaces. Philos.~Trans.~Roy.~Soc. London A 308 (1982)
523--615.

\bibitem[B1]{B1}P.~Boalch, ``Geometry and Braiding of Stokes Data;
Fission and Wild Character Varieties,'' Annals of Math.~\textbf{179}
(2014) 301--365. http://annals.math.princeton.edu/wp-content/uploads/annals-v179-n1-p05-s.pdf

\bibitem[B2]{B2}P.~Boalch, ``Hyperkaehler Manifolds and Nonabelian
Hodge Theory of (Irregular) Curves,'' arXiv:1203.6607

\bibitem[BB]{BB}O.~Biquard and P.~Boalch. Wild non-abelian Hodge
theory on curves - 2004, Compos.~Math.~140 (2004) 179--204.

\bibitem[BNR]{BNR}A.~Beauville, M.~S.~Narasimhan, and S.~Ramanan,
``Spectral curves and the generalized theta divisor,'' J.~reine
angew.~Math. 398 (1989) 169--179 URL: http://math1.unice.fr/\textasciitilde{}beauvill/pubs/bnr.pdf

\bibitem[Br]{Br}J.-L.~Brylinski, ``Loop spaces, characteristic classes
  and geometric quantization.'' Vol. 107. Springer Science \& Business Media, 2007.

\bibitem[D]{D}S.~K.~Donaldson, ``A new proof of a theorem of Narasimhan
and Seshadri,'' J.~Differential Geometry, \textbf{18} (1983) 269--277.

\bibitem[DG]{DG}R.~Y.~Donagi, and D.~Gaitsgory, ``The gerbe of Higgs
  bundles.'' Transformation groups 7.2 (2002): 109-153.

\bibitem[DP]{DP}R.~Donagi, and T.~Pantev, ``Langlands duality for
  Hitchin systems.'' arXiv preprint math/0604617 (2006).

\bibitem[DM]{DM}R.~Donagi and E.~Markman, ``Spectral covers, algebraically
completely integrable Hamiltonian systems, and moduli of bundles.''
arXiv:alg-geom/9507017

\bibitem[FG]{FG}V.~Fock and A.~Goncharov, Moduli spaces of local
systems and higher Teichmueller theory, Publ. Math. Inst. Hautes Etudes
Sci. No. 103 (2006), 1\textendash{}211.

\bibitem[GMN]{GMN} D.~Giaotto, G.~Moore, and A.~Neitzke, Wall-crossing,
Hitchin systems, and the WKB approximation. arXiv:0907.3987

\bibitem[HT]{HT}T.~Hausel and M.~Thaddeus. Mirror symmetry, Langlands
duality, and the Hitchin system. Invent. Math., 153 (1):197\textendash{}229,
2003.

\bibitem[H1]{H1}N.~Hitchin, ``The self-duality equations on a Riemann
surface,'' Proc. Long. Math. Soc. 55 (1987) 59--126.

\bibitem[H2]{H2}N.~Hitchin, ``Stable bundles and integrable systems,''
Duke Math J. 54 (1987) 91--114.

\bibitem[H3]{H3}N.~Hitchin, ``Lie Groups and Teichmueller Space,''
Topology 31 (1992) 449--473.

\bibitem[H4]{H4}N.~Hitchin, ``Moduli space of special Lagrangian
  submanifolds,'' arXiv preprint dg-ga/9711002 (1997).

\bibitem[H5]{H5}N.~Hitchin, ``Lectures on special Lagrangian
  submanifolds.'' AMS IP Studies in Advanced Mathematics 23 (2001):
  151-182.

\bibitem[HKLR]{HKLR}
  N.~J.~Hitchin, A.~Karlhede, U.~Lindstr\"om, M. Ro\v{c}ek,
  ``Hyperk\"ahler Metrics and Supersymmetry,''
  Commun. Math. Phys. 108 (1987) 535--589.
arXiv:0710.0631

\bibitem[KW]{KW}A.~Kapustin and E.~Witten, ``Electric-Magnetic
Duality and the Geometric Langlands Program,'' CNTP 1 (2007()1--236.
arXiv:hep-th/0604151

\bibitem[N]{N}A.~Neitzke, ``Hitchin systems in N=2 field theory,''
https://www.ma.utexas.edu/users/neitzke/expos/hitchin-systems.pdf

\bibitem[NS]{NS}M.~S.~Narasimhan and C.~S.~Seshadri, `Stable
and unitary vector bundles on a compact Riemann surface,' Ann.~of
Math.~\textbf{82} (1965) 540--567.

\bibitem[Ne]{Ne}P.~E.~Newstead, ``Introduction to moduli problems and
orbit spaces,'' Vol. 51 of Tata Institute of Fundamental Research
Lectures on Mathematics and Physics. Tata Institute of Fundamental
Research, Bombay (1978).

\bibitem[P]{P}A.~Polishchuk, ``Abelian varieties, theta functions and
  the Fourier transform.'' Vol. 153. Cambridge University Press,
  2003.

\bibitem[SYZ]{SYZ}A.~Strominger, S.~-T.~Yau, and E.~Zaslow. `` Mirror
  symmetry is T-duality.'' Nuclear Physics B 479.1 (1996): 243-259.

\bibitem[W]{W}E.~Witten, ``Gauge Theory and Wild Ramification,''

\end{thebibliography}

\end{document}
