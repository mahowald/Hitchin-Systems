%% Matthew's talk #1
\chapter{Construction of the Hitchin Moduli Space}


This talk closely follows \cite{H1}. In this chapter, it is shown
that the Hitchin moduli space $\mathcal{M}_{H}$ is a smooth manifold
of dimension $12(g-1)$. We also prove that $\mathcal{M}_{H}$ is
equipped with a complete hyperkahler metric.


\section{Definitions}

Let $M$ be a Riemannian surface, $P\rightarrow M$ a principal $G=SO\left(3\right)$-bundle,
$V$ the associated rank 2 vector bundle, and $\mathcal{G}$ the group
of gauge transformations ($\mathcal{G}=Map\left(M,G\right)$). Recall
that for a connection $A$ on $P$ and $\Phi\in\Omega_{M}^{1,0}\left(adP\otimes\mathbb{C}\right)$,
the self-dual equations are
\begin{align*}
\tag{\ensuremath{\star}}F_{A}+\left[\Phi,\Phi^{*}\right] & =0,\\
\overline{\partial_{A}}\Phi & =0
\end{align*}


For a connection $A$, $u\in\mathcal{G}$ acts by
\[
u\left(A\right)=uAu^{-1}-\left(du\right)u^{-1}
\]
(where the covariant derivative is $\nabla_{A}=d+A\wedge$, i.e.,
\[
\nabla_{u\left(A\right)}s=u\nabla_{A}\left(u^{-1}s\right)
\]
for a section $s$ of $V$.

For $\Phi\in\Omega_{M}^{1,0}\left(adP\otimes\mathbb{C}\right)$, $u$
acts by
\[
u\left(\Phi\right)=u\Phi u^{-1},
\]
where we regard $adP\otimes\mathbb{C}$ as the bundle of trace-zero
endomorphisms of $V$.
\begin{defn}
The Hitchin moduli space is
\[
\mathcal{M}_{H}:=\left\{ \mbox{solutions to (}\star\mbox{)}\right\} /\mathcal{G}.
\]
\end{defn}
\begin{description}
\item [{Goal}] To learn about the geometry and topology of $\mathcal{M}_{H}$.
\end{description}

\section{Summary of results}

Let $V$ be a rank 2, odd degree vector bundle over a Riemannian surface
$M$. Then,
\begin{itemize}
\item $\mathcal{M}_{H}$ is a smooth manifold of dimension $12\left(g-1\right)$.
\item $\mathcal{M}_{H}$ has a natural metric.
\item $\mathcal{M}_{H}$'s metric is complete and hyperkahler (in fact,
$\mathcal{M}_{H}$ is a hyperkahler quotient).\end{itemize}
\begin{rem*}
$V$ has odd degree $\implies$ there are no reducible solutions to
$\left(\star\right)$ $\implies$ $\mathcal{G}$ acts freely on $\left(\star\right)$.
\end{rem*}

\section{$\mathcal{M}_{H}$ is a dimension $12\left(g-1\right)$ smooth manifold}

First, to get the expected dimension, we compute the dimension of
the tangent space to $\mathcal{M}_{H}$ at a regular point (i.e.,
one with trivial isotropy group).


\subsection{Idea of Proof}
\begin{itemize}
\item Linearize $(\star)$ to determine the expected dimension.
\item Let $\left(A\times\Omega\right)_{0}$ denote ``regular points,''
i.e., ones which are only fixed by the identity in $\mathcal{G}$
(those with trivial isotropy group). Then, exhibit $\mathcal{M}_{H}$
is a smooth submanifold of $\left(\mathcal{A}\times\Omega\right)_{0}/\mathcal{G}$
using the regular value theorem.
\end{itemize}

\subsection{Linearization of $(\star)$}

Let $\left(\dot{A},\dot{\Phi}\right)\in\Omega_{M}^{1}\left(adP\right)\oplus\Omega_{M}^{1,0}\left(adP\otimes\mathbb{C}\right)$.
To get the linearization of $(\star$), fix a base point $\left(A,\Phi\right)$
and look at
\[
\left.\frac{d}{dt}\right|_{t=0}\left(\star\right)\left(A+t\dot{A},\Phi+t\dot{\Phi}\right).
\]
Recalling that $F_{A}=dA+A\wedge A$, the first equation becomes
\[
d_{A}\dot{A}+\left[\dot{\Phi},\Phi^{*}\right]+\left[\Phi,\dot{\Phi}^{*}\right]=0,
\]
and the second is
\[
\overline{\partial_{A}}\dot{\Phi}+\dot{A}^{0,1}\wedge\Phi=0.
\]


$\left(\dot{A},\dot{\Phi}\right)$ arises from an infinitesimal gauge
transformation $\dot{\psi}\in\Omega_{M}^{0}\left(adP\right)$ if
\begin{align*}
\dot{A} & =d_{A}\dot{\psi}, & \dot{\Phi} & =\left[\Phi,\dot{\psi}\right].
\end{align*}
Let
\begin{eqnarray*}
d_{1}:\Omega_{M}^{0}\left(adP\right) & \longrightarrow & \Omega_{M}^{1}\left(adP\right)\oplus\Omega_{M}^{1,0}\left(adP\otimes\mathbb{C}\right)\\
\dot{\psi} & \mapsto & \left(d_{A}\dot{\psi},\left[\Phi,\dot{\psi}\right]\right)
\end{eqnarray*}
and
\begin{eqnarray*}
d_{2}:\Omega_{M}^{1}\left(adP\right)\oplus\Omega_{M}^{1,0}\left(adP\otimes\mathbb{C}\right) & \longrightarrow &
\Omega_{M}^{2}\left(adP\right)\oplus\Omega_{M}^{2}\left(adP\otimes\mathbb{C}\right)\\
\left(\dot{A},\dot{\Phi}\right) & \mapsto &
\left(d_{A}\dot{A}+\left[\dot{\Phi},\Phi^{*}\right]+\left[\Phi,\dot{\Phi}^{*}\right],\overline{\partial_{A}}\dot{\Phi}+\dot{A}^{0,1}\wedge\Phi\right).
\end{eqnarray*}
Then, $d_{1},d_{2}$ define an \emph{elliptic complex} with index
$3\left(2-2g\right)-6\left(g-1\right)=12\left(1-g\right)$. The Atiyah-Singer
index theorem then says that
\[
\dim H^{0}-\dim H^{1}+\dim H^{2}=12\left(1-g\right).
\]



\subsection{Elliptic Complexes}

Now, a short digression into elliptic complexes. Let $X$ be a manifold,
$\pi:T^{*}X\rightarrow X$ be projection onto the base, and $\left\{ E_{k}\rightarrow X\right\} $
a collection of vector bundles over $X$.
\begin{defn}
A chain complex
\[
\xymatrix{\cdots\ar[r] & \Gamma\left(E_{k-1}\right)\ar[r]^{P_{k-1}} & \Gamma\left(E_{k}\right)\ar[r]^{P_{k}} & \Gamma\left(E_{k+1}\right)\ar[r] & \cdots}
\]
is \textbf{elliptic} if the corresponding sequence of symbols
\[
\xymatrix{\cdots\ar[r] & \pi^{*}E_{k-1}\ar[r]^{\sigma\left(P_{k-1}\right)} & \pi^{*}E_{k}\ar[r]^{\sigma\left(P_{k}\right)} & \pi^{*}E_{k+1}\ar[r] & \cdots}
\]
is exact.\end{defn}
\begin{example*}
Let $P:E_{1}\rightarrow E_{2}$ be a differential operator. Then,
the complex
\[
\xymatrix{0\ar[r] & \Gamma\left(E_{1}\right)\ar[r]^{P} & \Gamma\left(E_{2}\right)\ar[r] & 0}
\]
is elliptic means that
\[
\xymatrix{0\ar[r] & \pi^{*}E_{1}\ar[r]^{\sigma\left(P\right)} & \pi^{*}E_{2}\ar[r] & 0}
\]
is exact, i.e., that $\sigma\left(P\right):\pi^{*}E_{1}\rightarrow\pi^{*}E_{2}$
is an isomorphism. Recall that this is the ``usual'' definition
of elliptic differential operator.
\end{example*}
Now, return to our elliptic complex.

\[
\xymatrix{\Omega_{M}^{0}\left(adP\right)\ar[r]^-{d_{1}} & \Omega_{M}^{1}\left(adP\right)\oplus\Omega_{M}^{1,0}\left(adP\otimes\mathbb{C}\right)\ar[r]^-{d_{2}} &
\Omega_{M}^{2}\left(adP\right)\oplus\Omega_{M}^{2}\left(adP\otimes\mathbb{C}\right)\\
\dot{\psi}\ar@{|->}[r] & \left(d_{A}\dot{\psi},\left[\Phi,\dot{\psi}\right]\right)\\
 & \left(\dot{A},\dot{\Phi}\right)\ar@{|->}[r] &
 \left(d_{A}\dot{A}+\left[\dot{\Phi},\Phi^{*}\right]+\left[\Phi,\dot{\Phi}^{*}\right],\overline{\partial_{A}}\dot{\Phi}+\dot{A}^{0,1}\wedge\Phi\right)
}
\]

By construction, we're interested in $H^{1}$ of this complex: $H^{0}=\ker d_{1}$
is the covariantly constant $\dot{\psi}$ which commute with $\Phi$.
These correspond to reducible solutions to $(\star$). So, if $H^{0}\neq0$,
then $\left(A,\Phi\right)$ is reducible. In fact, by considering
$d_{2}^{*}$, we can show that $H^{2}=0$ as well \cite{H1}. Hence,
\[
-\dim H^{1}=12\left(1-g\right),
\]
i.e., for a regular point $\left(A,\Phi\right)$,
\[
\boxed{\dim T_{\left(A,\Phi\right)}\mathcal{M}_{H}=12\left(g-1\right)}.
\]



\subsection{$\mathcal{M}_{H}$ is a smooth manifold}

This is a sketch, for complete details see \cite{H1}.
\begin{defn}
$\left(A,\Phi\right)$ is a \textbf{regular point} if the isotropy
group of $\left(A,\Phi\right)$ is the identity (i.e., there are no
nontrivial gauge transformations fixing this point).
\end{defn}
Recall that infinitesimally, these are the points where there are
no nonzero solutions to $d_{1}\dot{\psi}=0$.

Let $\left(\mathcal{A}\times\Omega\right)_{0}$ denote the open set
of regular points in $\mathcal{A}\times\Omega$, and
\[
B:=\left(\mathcal{A}\times\Omega\right)_{0}/\mathcal{G}.
\]
By construction, $B$ is a Banach manifold with the quotient topology.
We have the map
\[
d_{1}^{*}:\Omega_{M}^{1}\left(adP\right)\oplus\Omega_{M}^{1,0}\left(adP\otimes\mathbb{C}\right)\longrightarrow\Omega_{M}^{0}\left(adP\right).
\]
Define a \emph{slice} of $\mathcal{M}_{H}$ to be $\ker d_{1}^{*}$,
at some fixed $\left(A_{0},\Phi_{0}\right)$---then, the slices provide
coordinate patchs for $B$. Set
\begin{eqnarray*}
f:\ker d_{1}^{*} & \longrightarrow & \Omega_{M}^{2}\left(adP\right)\oplus\Omega_{M}^{2}\left(adP\otimes\mathbb{C}\right)\\
\left(A,\Phi\right) & \mapsto & \left(F_{A}+\left[\Phi,\Phi^{*}\right],\overline{\partial_{A}}\Phi\right);
\end{eqnarray*}
then, $f^{-1}\left(0,0\right)$ is a smooth submanifold of $\ker d_{1}^{*}$
with dimension $12(g-1)$.

Since $\ker d_{1}^{*}$ form coordinate patches for $B$, the remainder
of the proof is just arguing that the $\ker d_{1}^{*}$ patch together
to form a smooth manifold.


\section{The tangent space}

Thanks to our results in the previous section, we have an explicit
description of the tangent space to $\mathcal{M}_{H}$:
\[
T_{\left(A,\Phi\right)}\mathcal{M}_{H}=\left\{
\left(\dot{A},\dot{\Phi}\right)\left|\begin{matrix}d_{A}\dot{A}+\left[\dot{\Phi},\Phi^{*}\right]+\left[\Phi,\dot{\Phi}^{*}\right]=0,\\
\overline{\partial_{A}}\dot{\Phi}+\dot{A}^{0,1}\wedge\Phi=0,\\
d_{A}^{*}\dot{A}+Re\left[\Phi^{*},\dot{\Phi}\right]=0.
\end{matrix}\right.\right\}
\]
The third equation appears because $d_{1}^{*}\left(\dot{A},\dot{\Phi}\right)=d_{A}^{*}\dot{A}+Re\left[\Phi^{*},\dot{\Phi}\right]$.


\section{$\mathcal{M}_{H}$ has a complete hyperkahler metric}

Recall from Sean's talk that the metric on $\mathcal{A}\times\Omega$
given by
\[
g\left((\psi,\phi),\left(\psi,\phi\right)\right)=2i\int_{M}Tr\left(\psi^{*}\psi+\phi\phi^{*}\right)
\]
induces a metric on $T_{p}\mathcal{M}_{H}$. We want to show that
this is a complete metric on $T_{\left(A,\Phi\right)}\mathcal{M}_{H}$.
As a reminder:
\begin{defn*}
A metric on $M$ is \textbf{complete} if every Cauchy sequence of
points on $M$ has a limit which is also in $M$.
\end{defn*}

\subsection{Idea of Proof}

By contradiction: Suppose we have a sequence of points in $\mathcal{M}_{H}$
defined by a geodesic $\gamma$ converging to a point not in $\mathcal{M}_{H}$.
Because $g$ is $\mathcal{G}$-invariant, we can look at $\tilde{\mathcal{M}_{H}}\subset\mathcal{A}\times\Omega$.
Lift $\gamma$ to a horizontal $\tilde{\gamma}$ in $\tilde{\mathcal{M}}_{H}$.
$\tilde{\gamma}$ still defines a Cauchy sequence in $\tilde{\mathcal{M}}_{H}$,
so we have
\[
||A_{n}-\overline{A}||_{L^{2}}^{2}+||\Phi_{n}-\overline{\Phi}||_{L^{2}}^{2}\leq C
\]
 for some $C$ as $\left(A_{n},\Phi_{n}\right)\rightarrow\left(\overline{A},\overline{\Phi}\right)$
(where $\left(\overline{A},\overline{\Phi}\right)$ is the limiting
point not in $\mathcal{M}_{H}$).

Now, we apply Uhlenbeck's compactification theorem to show that there's
a gauge transformation taking $\left(\overline{A},\overline{\Phi}\right)$
to a solution of $(\star)$:
\begin{thm}
[Uhlenbeck] There are constants $\epsilon_{1}$, $M>0$ such that
any connection $A$ on the trivial bundle over $\overline{B}^{4}$
with $||F_{A}||_{L^{2}}<\epsilon_{1}$ is gauge equivalent to a connection
$\tilde{A}$ over $B^{4}$ with
\begin{enumerate}
\item $d^{*}\tilde{A}=0$,
\item $\lim_{\left|x\right|\rightarrow1}\tilde{A}_{r}=0$, and
\item $||\tilde{A}||_{L_{1}^{2}}\leq M||F_{\tilde{A}}||_{L^{2}}$.
\end{enumerate}
Moreover for suitable constants $\epsilon_{1}$, $M$, $\tilde{A}$
is uniquely determined by these properties, up to $\tilde{A}\mapsto u_{0}\tilde{A}u_{0}^{-1}$
for a constant $u_{0}$ in $U\left(n\right)$.
\end{thm}
In particular, we can use the following corrollary:
\begin{cor*}
For any sequence of ASD connections $A_{\alpha}$ over $\overline{B}^{4}$
with $||F\left(A_{\alpha}\right)||_{L^{2}}\leq\epsilon$, there is
a subsequence $\alpha'$ and gauge equivalent connections $\tilde{A}_{\alpha'}$
which converge in $C^{\infty}$ on the open ball.
\end{cor*}
Therefore, there's a gauge transformation taking $\left(\overline{A},\overline{\Phi}\right)$
to a solution of $(\star)$. This is a contradiction, so $\mathcal{M}_{H}$
is complete.


\subsection{Hyperkahler structure}

Recall from Sean's talk that there's a symplectic form on $\mathcal{M}_{H}$
given by
\[
\omega\left((\psi_{1},\phi_{1}),\left(\psi_{2},\phi_{2}\right)\right)=\int_{M}Tr\left(\phi_{2}\psi_{1}-\phi_{1}\psi_{2}\right).
\]
This defines a complex moment map from the action of $\mathcal{G}$:
\[
\mu\left(A,\Phi\right)=\overline{\partial}_{A}\Phi.
\]
We can write $\mu=\mu_{2}+i\mu_{3}$ and $\omega=\omega_{2}+i\omega_{3}$
to get two symplectic structures out of this. The third (or first?)
symplectic structure is just the Kahler form associated to the metric
\[
g=2i\int_{M}Tr\left(\psi^{*}\psi+\phi\phi^{*}\right),
\]
and has moment map
\[
\mu_{1}\left(A,\Phi\right)=F_{A}+\left[\Phi,\Phi^{*}\right].
\]
This exhibits $\mathcal{M}_{H}$ as a hyperkahler quotient of $\mathcal{A}\times\Omega$:
\[
\mathcal{M}_{H}=\bigcap_{i=1}^{3}\mu_{i}^{-1}\left(0\right)/\mathcal{G}.
\]



\section{Summary of topological results}

Here, we state some topological results. For proofs, see section 7
of \cite{H1}.

$\mathcal{M}_{H}$ is...
\begin{itemize}
\item non-compact
\item connected and simply connected
\item the Betti numbers $b_{i}$ vanish for $i>6g-6.$
\end{itemize}