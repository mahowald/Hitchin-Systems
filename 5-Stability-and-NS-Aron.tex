%!TEX root = HitchinSystems.tex
\newcommand{\sA}{\mathcal{A}}
\newcommand{\sB}{\mathcal{B}}
\newcommand{\sE}{\mathcal{E}}
\newcommand{\sF}{\mathcal{F}}
\newcommand{\sG}{\mathcal{G}}
\newcommand{\sL}{\mathcal{L}}
\newcommand{\sO}{\mathcal{O}}
\newcommand{\sP}{\mathcal{P}}

\newcommand{\bZ}{\mathbb{Z}}
\newtheorem{lem}{Lemma}

\chapter{Stability and the Narasimhan-Seshadri Theorem}

In this chapter we will give an idea of how the classical theorem of
Narasimhan-Seshadri is proved. We follow mainly \cite{NS}. It states
the following:
\begin{thm}
Let $X$ be a compact Riemann surface, of genus $\geq 2$ then the
isomorphism classes\footnote{This means: representations are considered up to
  conjugation, algebraic vector bundles up to isomorphism of vector
  bundles and connections up to gauge fixing the imposed equation.} of
the following three sets are in bijection:
\begin{enumerate}[(i)]
\item $n$-dimensional irreducible unitary representations of
  $\pi_1(X)$;
\item stable rank $n$, degree $0$, algebraic vector bundles over $X$;
\item indecomposable rank $n$ holomorphic vector bundles with an
  unitary connection $A$, s.t. $F_A=0$\footnote{Here we denote by
    $F_A$ the curvature of the connection $A$.}.
\end{enumerate}
\end{thm}

\begin{rem*}
Actually Narasimhan-Seshadri result is more general. They consider a cover $p: Y
\rightarrow X$ ramified of degree $d \in [-n+1,0]$ at a single point $x_0 \in
X$, s.t. $Y/H \simeq X$ and $Y$ is simply connected. One modifies the
above classes as follows:
\begin{enumerate}[(i)']
\item homomorphisms $\rho: H \rightarrow U(n)$ of type $\xi_{-d}$,
  i.e. $\left.\rho\right|_{\mbox{Stab}_{x_0}}=\zeta_{-d}$\footnote{The
  stabilizer of any pre-image of $x_0$, which is cyclic of order $n$,
  acts by multiplication by a root of unity.};
\item stable rank $n$, degree $d$, algebraic vector bundles over $X$;
\item indecomposable rank $n$ holomorphic vector bundles over $X$, with an
  unitary connection $A$, s.t. $F_A=-2\pi i\frac{d}{n}$.
\end{enumerate}
\end{rem*}

\begin{rem*}
Stricly speaking Narasimhan-Seshadri theorem is only the equivalence
between (i)' and (ii)' above. A newer proof by Donaldson
\cite{D} stablishes the relation between (ii)' and
(iii)'\footnote{The relation between (iii)' and (i)' is an easy case
  of Riemann-Hilbert correspondence. For the degree $0$ case this can
  be seem directly.}. More importantly, their proof proceeds by
induction on the rank and they stablish the following result as well,
which is essential to make the inductive step work. Let $T$ be an
algebraic variety and $\left\{ E_t\right\}_{T}$ be a flat family of
algebraic vector bundles over $X$. Then $T_s \subset T$, the set of
points $t$, s.t. $E_t$ is stable, is Zariski open. Similarly for
semistable.
\end{rem*} 

\begin{rem*}
The relevance of this result to this class is that Hitchin proves a
more general result \cite[Proposition 3.4]{H1} by yet a different
method. Taking $\Phi=0$ recovers the case $n=2$ above. He remarks in
the introduction that one should be able to obtain the result for
arbitrary $n$, by considering $GL_n$. (I don't know of a reference for
this.)
\end{rem*}

From now one we will always be working in the algebraic category, that
is by vector bundle we will mean an algebraic vector bundle, and
families are algebraic families and so forth. We will also restrict to
the simpler form we first stated and give the elements of the proof for
it, the more general case is not much harder.

\section{Weil's theorem}

Firstly, we need to say a little bit about how to construct a vector
bundle from a representation of $\pi_1(X)$, and which ones come from
these. This is essentially the following result of Weil.

\begin{prop} 
An indecomposable vector $E$ over $X$ comes from a representation of
$\pi_1(X)$ if and only if $c_1(E)=0$.
\end{prop}

\begin{proof}
One direction is a construction. Let $p: Y \rightarrow X$ be a simply connected covering
of $X$, and $\rho: \pi_1(X) \rightarrow GL(V)$ a representation, then define 
\[\Phi(\rho) \equiv p_{*}\left(Y\times_{\pi_1(X)}V\right)^{\pi_1(X)},\]
where we confund $Y\times_{\pi_1(X)}V$ the associated vector bundle over $Y$
with the corresponding locally constant sheaf\footnote{The superscript $\pi_1(X)$
means we consider the invariant part with respect to the $\pi_1(X)$ action by
pullback.}. If $\dim(V) = n$ then $\Phi(\rho)$ is locally constant of
rank $n$.

For the converse, suppose $E$ is an indecomposable vector bundle
over $X$. Let $P$ denote the associated principal $GL_n$-bundle, then
one has the following exact sequence:
\[0 \rightarrow \mbox{ad}(P) \rightarrow TP^{GL_n} \rightarrow TX \rightarrow 0,\]
whose splitting is the data of an algebraic connection on $P$\footnote{The
middle term is the invariant vectors of $TP$ with respect to the
action of $GL_n$ induced from the $GL_n$-bundle structure, and
$\mbox{ad}(P)$ is the associated vector bundle to $P$ obtained by
taking the adjoint representation of $GL_n$ on its Lie algebra.}. There
exists an element $\mbox{at}(P) \in H^1(X,\mbox{Hom}(TX,\mbox{ad}(P)))$ which
measues when $P$ has a connection\footnote{This is the so-called
  Atiyah class of a principal $G$-bundle and the preceeding exact
  sequence defines the Atiyah (Lie) algebroid.}.
Then one has the following non-trivial fact \cite[Section 5]{A}:
\begin{lem}
If $E$ is indecomposable then $\mbox{at}(E) = c_1(E)$.
\end{lem}
So by assumption our $E$ admits an algebraic connection
$\nabla: E \rightarrow E\otimes\Omega^1_X$. Moreover since $X$ has complex
dimension 1, $\nabla$ is integrable. So $\ker \nabla$ is a locally
constant sheaf of vector spaces over $X$, i.e. a representation of
$\pi_1(X)$.
\end{proof}

A not so hard fact to check is that $\Phi$ is actually a
functor\footnote{From the category of finite dimensional
  representations of $\pi_1(X)$ to that of vector bundles over $X$.}, that is
\[\Phi\left(\rho,\rho'\right) = \mathcal{H}\mbox{om}_X(\Phi(\rho),\Phi(\rho')).\]

The important result now is:
\begin{lem} 
When restricted to unitary representations $\Phi$ is fully faithful,
i.e.
\[\mbox{Hom}(\rho,\rho') \simeq \mathcal{H}\mbox{om}_X(\Phi(\rho),\Phi(\rho')).\]
\end{lem}

\begin{proof}
Since the category of finite dimensional representations is
semisimple, it is enough to consider irreducible objects, so one can
suppose $(\rho,V)$ is trivial, then the statement becomes.
\[V'^{\pi_1(X)} \simeq H^0(\pi_1(X),V') \simeq
H^0(X,\Phi(V')),\]
for any unitary finite dimensional represetation $V'$.
Given $v \in V'^{\pi_1(X)}$ one has a section $s(y) = (y,v)$ of
$Y\times_HV$, which is $\pi_1(X)$-equivariant so it gives a section of
$\Phi(\rho')$. This clearly is zero only if $v$ is zero so that is
injective. 
Now for the surjectivity suppose one has a section $f \in
H^0(X,\Phi(\rho'))$, this is a function $\tilde{f}: Y \rightarrow V$, s.t.
\[\tilde{f}\circ h = \rho'(h) \tilde{f}.\]
Since $V$ is unitary, let $||\tilde{f}||^2 =
\left<\tilde{f},\tilde{f}\right>_{V}$. By definition
$||\tilde{f}||^2:Y \rightarrow \mathbb{C}$, with $||\tilde{f}||^2\circ h =
||\tilde{f}||^2$, so one gets a globally defined function on
$X$. Since $X$ is projective this function is constant. Now
$||\tilde{f}||^2$ constant implies that $\tilde{f}$ is also constant,
so $\rho'$ acts trivially, or $\tilde{f}$ factor through
$V'^{\pi_1(X)} \rightarrow V$.
\end{proof}

\section{Proof of main theorem}

As we mentioned before the proof proceeds by induction on two assetion
let's recall them for convinience.

\begin{thm}
Let $X$ be a compact Riemann surface of genus $\geq 2$.
\begin{enumerate}[(i)]
\item A rank $n$ vector bundle $E$ over $X$ of degree 0 is stable if and only if it
is given by $\Phi(\rho)$ for some homomorphism $\rho: \pi_1(X) \rightarrow
U(n)$, i.e. an unitary representation of $\pi_1(X)$.
\item For any family $T$ of vector bundles over $X$, the subset $T_s
  \subset T$ (resp. $T_{ss}$) of points whose fiber is a stable
  (resp. semistable) vector bundle is Zariski open.
\end{enumerate}
\end{thm}

We will essentially just prove (i). To prove (ii) one needs to
consider specific models for the moduli spaces considered above and
analise what the stability (or semistability) condition is inside
them. This is standard depending on how one construct these moduli
spaces.

\begin{proof}
\textit{Step 1 (line bundles):} The statement (ii) is trivial as stability is a
vacuuos condition for line bundles.
For (i) one recalls another definition of $c_1$ for line
bundles. Consider the exponential exact sequence\footnote{Strictly
  speaking, we use GAGA to be able to talk about the exponential map.}
of sheaves
\[0 \rightarrow \underline{\bZ} \rightarrow \sO_X \rightarrow \sO^{\times}_X 
\rightarrow 0.\]
Then one has a long exact sequence in cohomology
\[\cdots \rightarrow H^1(X;\bZ) \rightarrow H^1(X,\sO_X) \rightarrow H^1(X,\sO^{\times}_X)
\overset{c_1}{\rightarrow} H^2(X;\bZ) \rightarrow \cdots.\]
This gives that $c^{-1}_1(0) \simeq H^1(X;U(1)) \simeq
\mbox{Hom}(H_1(X;\bZ),U(1)) \simeq \mbox{Hom}(\pi_1(X),U(1))$\footnote{Here
  $H^k(X;-)$ denotes singular cohomology and $H^k(X,-)$ denotes
  coherent cohomology.}.

\textit{Step 2 (irreducible unitary representations exist):} Since
$\pi_1(X)$ could have only trivial unitary representations one needs some
calculation to show that the space of such is non-empty. For example
for $g=2$ (genus of X), one has generators $A_1,A_2,B_1,B_2$ for
$\pi_1(X)$ and one can take $B_1 =
\mbox{diag}(\xi_N,\xi^2_N,\ldots,\xi^N_N)$, where $\xi_N$ is the $N$th
root of unity, $A_2$ the matrix with $1$'s only in the diagonal above
the main diagonal, and $A_1=B^{-1}_1$ and $B_2 = A^{-1}_2$. A similar
construction works for $g > 2$ as well.

\textit{Step 3 (characterize the image of unitary and irreducible unitary
representations):} Let $\rho$ be an unitary representation, then
$\Phi(\rho)$ is semistable. Indeed, suppose there is $F \subset
\Phi(\rho)$, s.t. $\deg(F) \geq 0$, we will prove that $\deg(F) =
0$. By contradiction, let $\deg(F) > 0$, by taking the exterior power
$\Lambda^{\mbox{rk}(F)}F \subset \Lambda^{\mbox{rk(F)}}\Phi(\rho)
\equiv \Phi(\rho')$ one can suppose that $F$ has rank 1. Now let
$\rho' = \oplus \sigma_i$ with $\sigma_i$ irreducible, we have
projections:
\[p_i: \Phi(\rho') \rightarrow \Phi(\sigma_i).\]
By tensoring with degree $0$ line bundle $\sL$ one can suppose $F =
\sO_X(dx_0)$ for some $x_o \in X$ and $d>0$, notice that for the
representation $\rho'$ this just means tensoring with a character
$\xi_{\sL}$. Now if $\sigma_0$ is the trivial representation than
$\left.p_0\right|_F$ is zero, because $\mbox{deg}(\Phi(\sigma_0))$ and a
non-zero map between line bundles can not decrease the
degree\footnote{One sees this easily by taking the definition of
  degree as the number of zeros minus the number of poles of a rational section.}. Now
$H^0(X,F)\neq 0$, because $F=\sO_X(dx_0)$, however
$H^0(X,\Phi(\sigma_i)) = 0$ for any non-trivial irreducible
$\sigma_i$, by $H^0(X,\Phi(\sigma_i)) \simeq \sigma^{\pi_1(X)}_i = 0$
since $\sigma_i$ is 1-dimensional. This gives a contradiction with
non-zero $p_i$'s, since $F$ is a subbundle of $\Phi(\rho')$ it most be
trivial, which is a contradiction with $\mbox{deg}(F)>0$.
Notice that by using the induction hypothesis for smaller
representations and the result just proved for the semistable one
obtains that if $\rho$ is an irreducible unitary representation than
$\Phi(\rho)$ is stable.

\textit{Step 4 (the variety of representations):} One has a structure of
algebraic variety on the space of representations of a fixed dimension of
$\pi_1(X)$. Indeed these are given by a collection of matrices, $2g$ of
them, satisfying one algebraic equation. Now this gives an algebraic
space parametrizing vector bundles over $X$ obtained applying the
functor $\Phi$. Moreover, one can put also an algebraic structure on
the space of rank $n$ degree $d$ vector bundles over $X$. The set of indecomposable vector bundles inside these
forms an irreducible variety \footnote{Both of these statements follow
  form the representability of the Quot scheme, though \cite{NS} gives
  a more direct argument.}. By Weil's theorem this is equivalent to
the subvariety $S$ of $n$-dimensional representation whose associated
vector bundle is stable.

\textit{Step 5 (surjectivity):} Let $S$ be the subset of $n$-dimensional
representations such that $\Phi(S)$ is stable. Then $S$ is a Zariski
open set (assuming (ii)) of an irreducible variety (Step 4), hence
connected. Now let $U_0 = U\cap S$ where $U$ is the set of unitary
representations, and $U_0$ is the set of irreducible and unitary
representations (Step 3). So $U_0$ is open and closed inside of $S$,
which is connected, since $U_0$ is not empty (Step 2) $U_0=S$. This
finishes the proof.
\end{proof}
